%!TEX root = ts.tex

\rSec0[general]{General Principles}

\rSec1[conformance]{Conformance}

\pnum
 Conformance is specified in terms of behavior. Ideal behavior is not always implementable, so the conformance subclauses take that into account.


\rSec2[conformance.9945]{POSIX conformance}

\pnum
 Some behavior is specified by reference to POSIX. How such behavior is actually implemented is unspecified.

\pnum
 \begin{note} This constitutes an ``as if'' rule allowing implementations to call native operating system or other APIs. \end{note}

\pnum
Implementations are encouraged to provide such behavior as it is defined by POSIX. Implementations shall document any behavior that differs from the behavior defined by POSIX. Implementations that do not support exact POSIX behavior are encouraged to provide behavior as close to POSIX behavior as is reasonable given the limitations of actual operating systems and file systems. If an implementation cannot provide any reasonable behavior, the implementation shall report an error as specified in Error Reporting~(\ref{err.report}).

\pnum
 \begin{note} This allows users to rely on an exception being thrown or an error code being set when an implementation cannot provide any reasonable behavior. \end{note}

\pnum
 Implementations are not required to provide behavior that is not supported by a particular operating system.



\rSec2[conformance.conditional]{Conditionally-supported features}

\pnum
This document defines conditionally-supported features, in the form of additional member functions on types that satisfy \tcode{Protocol}~(\ref{socket.reqmts.protocol}), \tcode{Endpoint}~(\ref{socket.reqmts.endpoint}), \tcode{SettableSocketOption}~(\ref{socket.reqmts.settablesocketoption}), \tcode{GettableSocketOption}~(\ref{socket.reqmts.gettablesocketoption}) or \tcode{IoControlCommand}~(\ref{socket.reqmts.iocontrolcommand}) requirements.

\pnum
 \begin{note} This is so that, when the additional member functions are available, \Cpp programs can extend the library to add support for other protocols and socket options. \end{note}

\pnum
For the purposes of this document, implementations that provide all of the additional member functions are known as extensible implementations.

\pnum
 \begin{note} Implementations are encouraged to provide the additional member functions, where possible. It is intended that POSIX and Windows implementations will provide them. \end{note}


\rSec1[intro.ack]{Acknowledgments}

\pnum
The design of this specification is based, in part, on the Asio library
written by Christopher Kohlhoff.

