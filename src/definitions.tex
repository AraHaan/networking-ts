%!TEX root = ts.tex

\rSec0[defs]{Terms and definitions}

\pnum
\indextext{definitions|(}%
For the purposes of this document,
the terms and definitions
given in ISO/IEC 14882:2014,
ISO/IEC 2382-1:1993,
and the following apply.

\pnum
ISO and IEC maintain terminological databases
for use in standardization
at the following addresses:
\begin{itemize}
\item IEC Electropedia: available at \url{http://www.electropedia.org/}
\item ISO Online browsing platform: available at \url{http://www.iso.org/obp}
\end{itemize}

\pnum
Terms that are used only in a small portion of this document
are defined where they are used and italicized where they are
defined.

\indexdefn{host byte order}%
\definition{host byte order}{defs.host.byte.order}
the arrangement of bytes in any integer type when using a specific machine architecture

[SOURCE: ISO/IEC 9945:2009, 3.193]

\indexdefn{network byte order}%
\definition{network byte order}{defs.net.byte.order}
the way of representing any integer type such that, when transmitted over a network via a network endpoint,
the \tcode{int} type is transmitted as an appropriate number of octets with the most significant octet first,
followed by any other octets in descending order of significance

[SOURCE: ISO/IEC 9945:2009, 3.237]

\indexdefn{synchronous operation}%
\definition{synchronous operation}{defs.sync.op}
operation where control is not returned until the operation completes

\indexdefn{asynchronous operation}%
\definition{asynchronous operation}{defs.async.op}
operation where control is returned immediately without waiting for the operation to complete

\begin{defnote}Multiple asynchronous operations may be executed concurrently.\end{defnote}

\setcounter{tocdepth}{1}
