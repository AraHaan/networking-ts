%!TEX root = ts.tex

\rSec0[internet]{Internet protocol}


\indexlibrary{\idxhdr{experimental/internet}}%
\rSec1[internet.synop]{Header \tcode{<experimental/internet>} synopsis}

\indexlibrary{\idxcode{resolver_errc}}%
\indexlibrary{\idxcode{port_type}}%
\indexlibrary{\idxcode{scope_id_type}}%
\indexlibrary{\idxcode{v4_mapped_t}}%
\indexlibrary{\idxcode{v4_mapped}}%
\begin{codeblock}
namespace std {
namespace experimental {
namespace net {
inline namespace v1 {
namespace ip {

  enum class resolver_errc {
    host_not_found = @\EXPO{an \impldef{value of \tcode{resolver_errc::host_not_found}} non-zero value}@,   // \tcode{\EXPO{EAI_NONAME}}
    try_again = @\EXPO{an \impldef{value of \tcode{resolver_errc::try_again}} non-zero value}@,        // \tcode{\EXPO{EAI_AGAIN}}
    service_not_found = @\EXPO{an \impldef{value of \tcode{resolver_errc::service_not_found}} non-zero value}@ // \tcode{\EXPO{EAI_SERVICE}}
  };

  const error_category& resolver_category() noexcept;

  error_code make_error_code(resolver_errc e) noexcept;
  error_condition make_error_condition(resolver_errc e) noexcept;

  using port_type = uint_least16_t;
  using scope_id_type = uint_least32_t;

  struct v4_mapped_t {};
  constexpr v4_mapped_t v4_mapped;

  class address;
  class address_v4;
  class address_v6;

  class bad_address_cast;

  // \ref{internet.address.comparisons}, address comparisons:
  constexpr bool operator==(const address&, const address&) noexcept;
  constexpr bool operator!=(const address&, const address&) noexcept;
  constexpr bool operator< (const address&, const address&) noexcept;
  constexpr bool operator> (const address&, const address&) noexcept;
  constexpr bool operator<=(const address&, const address&) noexcept;
  constexpr bool operator>=(const address&, const address&) noexcept;

  // \ref{internet.address.v4.comparisons}, address_v4 comparisons:
  constexpr bool operator==(const address_v4&, const address_v4&) noexcept;
  constexpr bool operator!=(const address_v4&, const address_v4&) noexcept;
  constexpr bool operator< (const address_v4&, const address_v4&) noexcept;
  constexpr bool operator> (const address_v4&, const address_v4&) noexcept;
  constexpr bool operator<=(const address_v4&, const address_v4&) noexcept;
  constexpr bool operator>=(const address_v4&, const address_v4&) noexcept;

  // \ref{internet.address.v6.comparisons}, address_v6 comparisons:
  constexpr bool operator==(const address_v6&, const address_v6&) noexcept;
  constexpr bool operator!=(const address_v6&, const address_v6&) noexcept;
  constexpr bool operator< (const address_v6&, const address_v6&) noexcept;
  constexpr bool operator> (const address_v6&, const address_v6&) noexcept;
  constexpr bool operator<=(const address_v6&, const address_v6&) noexcept;
  constexpr bool operator>=(const address_v6&, const address_v6&) noexcept;

  // \ref{internet.address.creation}, address creation:
  address make_address(const char*);
  address make_address(const char*, error_code&) noexcept;
  address make_address(const string&);
  address make_address(const string&, error_code&) noexcept;
  address make_address(string_view);
  address make_address(string_view, error_code&) noexcept;

  // \ref{internet.address.v4.creation}, address_v4 creation:
  constexpr address_v4 make_address_v4(const address_v4::bytes_type&);
  constexpr address_v4 make_address_v4(address_v4::uint_type);
  constexpr address_v4 make_address_v4(v4_mapped_t, const address_v6&);
  address_v4 make_address_v4(const char*);
  address_v4 make_address_v4(const char*, error_code&) noexcept;
  address_v4 make_address_v4(const string&);
  address_v4 make_address_v4(const string&, error_code&) noexcept;
  address_v4 make_address_v4(string_view);
  address_v4 make_address_v4(string_view, error_code&) noexcept;

  // \ref{internet.address.v6.creation}, address_v6 creation:
  constexpr address_v6 make_address_v6(const address_v6::bytes_type&,
                                       scope_id_type = 0);
  constexpr address_v6 make_address_v6(v4_mapped_t, const address_v4&) noexcept;
  address_v6 make_address_v6(const char*);
  address_v6 make_address_v6(const char*, error_code&) noexcept;
  address_v6 make_address_v6(const string&);
  address_v6 make_address_v6(const string&, error_code&) noexcept;
  address_v6 make_address_v6(string_view);
  address_v6 make_address_v6(string_view, error_code&) noexcept;

  // \ref{internet.address.io}, address I/O:
  template<class CharT, class Traits>
    basic_ostream<CharT, Traits>& operator<<(
      basic_ostream<CharT, Traits>&, const address&);

  // \ref{internet.address.v4.io}, address_v4 I/O:
  template<class CharT, class Traits>
    basic_ostream<CharT, Traits>& operator<<(
      basic_ostream<CharT, Traits>&, const address_v4&);

  // \ref{internet.address.v6.io}, address_v6 I/O:
  template<class CharT, class Traits>
    basic_ostream<CharT, Traits>& operator<<(
      basic_ostream<CharT, Traits>&, const address_v6&);

  template<class> class basic_address_iterator; // \notdef
  template<> class basic_address_iterator<address_v4>;
  using address_v4_iterator = basic_address_iterator<address_v4>;
  template<> class basic_address_iterator<address_v6>;
  using address_v6_iterator = basic_address_iterator<address_v6>;

  template<class> class basic_address_range; // \notdef
  template<> class basic_address_range<address_v4>;
  using address_v4_range = basic_address_range<address_v4>;
  template<> class basic_address_range<address_v6>;
  using address_v6_range = basic_address_range<address_v6>;

  class network_v4;
  class network_v6;

  // \ref{internet.network.v4.comparisons}, network_v4 comparisons:
  bool operator==(const network_v4&, const network_v4&) noexcept;
  bool operator!=(const network_v4&, const network_v4&) noexcept;

  // \ref{internet.network.v6.comparisons}, network_v6 comparisons:
  bool operator==(const network_v6&, const network_v6&) noexcept;
  bool operator!=(const network_v6&, const network_v6&) noexcept;

  // \ref{internet.network.v4.creation}, network_v4 creation:
  network_v4 make_network_v4(const address_v4&, int);
  network_v4 make_network_v4(const address_v4&, const address_v4&);
  network_v4 make_network_v4(const char*);
  network_v4 make_network_v4(const char*, error_code&) noexcept;
  network_v4 make_network_v4(const string&);
  network_v4 make_network_v4(const string&, error_code&) noexcept;
  network_v4 make_network_v4(string_view);
  network_v4 make_network_v4(string_view, error_code&) noexcept;

  // \ref{internet.network.v6.creation}, network_v6 creation:
  network_v6 make_network_v6(const address_v6&, int);
  network_v6 make_network_v6(const char*);
  network_v6 make_network_v6(const char*, error_code&) noexcept;
  network_v6 make_network_v6(const string&);
  network_v6 make_network_v6(const string&, error_code&) noexcept;
  network_v6 make_network_v6(string_view);
  network_v6 make_network_v6(string_view, error_code&) noexcept;

  // \ref{internet.network.v4.io}, network_v4 I/O:
  template<class CharT, class Traits>
    basic_ostream<CharT, Traits>& operator<<(
      basic_ostream<CharT, Traits>&, const network_v4&);

  // \ref{internet.network.v6.io}, network_v6 I/O:
  template<class CharT, class Traits>
    basic_ostream<CharT, Traits>& operator<<(
      basic_ostream<CharT, Traits>&, const network_v6&);

  template<class InternetProtocol>
    class basic_endpoint;

  // \ref{internet.endpoint.comparisons}, basic_endpoint comparisons:
  template<class InternetProtocol>
    bool operator==(const basic_endpoint<InternetProtocol>&,
                    const basic_endpoint<InternetProtocol>&);
  template<class InternetProtocol>
    bool operator!=(const basic_endpoint<InternetProtocol>&,
                    const basic_endpoint<InternetProtocol>&);
  template<class InternetProtocol>
    bool operator< (const basic_endpoint<InternetProtocol>&,
                    const basic_endpoint<InternetProtocol>&);
  template<class InternetProtocol>
    bool operator> (const basic_endpoint<InternetProtocol>&,
                    const basic_endpoint<InternetProtocol>&);
  template<class InternetProtocol>
    bool operator<=(const basic_endpoint<InternetProtocol>&,
                    const basic_endpoint<InternetProtocol>&);
  template<class InternetProtocol>
    bool operator>=(const basic_endpoint<InternetProtocol>&,
                    const basic_endpoint<InternetProtocol>&);

  // \ref{internet.endpoint.io}, basic_endpoint I/O:
  template<class CharT, class Traits, class InternetProtocol>
    basic_ostream<CharT, Traits>& operator<<(
      basic_ostream<CharT, Traits>&,
      const basic_endpoint<InternetProtocol>&);

  template<class InternetProtocol>
    class basic_resolver_entry;

  template<class InternetProtocol>
    bool operator==(const basic_resolver_entry<InternetProtocol>&,
                    const basic_resolver_entry<InternetProtocol>&);
  template<class InternetProtocol>
    bool operator!=(const basic_resolver_entry<InternetProtocol>&,
                    const basic_resolver_entry<InternetProtocol>&);

  template<class InternetProtocol>
    class basic_resolver_results;

  template<class InternetProtocol>
    bool operator==(const basic_resolver_results<InternetProtocol>&,
                    const basic_resolver_results<InternetProtocol>&);
  template<class InternetProtocol>
    bool operator!=(const basic_resolver_results<InternetProtocol>&,
                    const basic_resolver_results<InternetProtocol>&);

  class resolver_base;

  template<class InternetProtocol>
    class basic_resolver;

  string host_name();
  string host_name(error_code&);
  template<class Allocator>
    basic_string<char, char_traits<char>, Allocator>
      host_name(const Allocator&);
  template<class Allocator>
    basic_string<char, char_traits<char>, Allocator>
      host_name(const Allocator&, error_code&);

  class tcp;

  // \ref{internet.tcp.comparisons}, tcp comparisons:
  bool operator==(const tcp& a, const tcp& b);
  bool operator!=(const tcp& a, const tcp& b);

  class udp;

  // \ref{internet.udp.comparisons}, udp comparisons:
  bool operator==(const udp& a, const udp& b);
  bool operator!=(const udp& a, const udp& b);

  class v6_only;

  namespace unicast {

    class hops;

  } // namespace unicast

  namespace multicast {

    class join_group;

    class leave_group;

    class outbound_interface;

    class hops;

    class enable_loopback;

  } // namespace multicast
} // namespace ip
} // inline namespace v1
} // namespace net
} // namespace experimental

  template<> struct is_error_condition_enum<experimental::net::v1::ip::resolver_errc>
    : public true_type {};

  // \ref{internet.hash}, hash support
  template<class T> struct hash;
  template<> struct hash<experimental::net::v1::ip::address>;
  template<> struct hash<experimental::net::v1::ip::address_v4>;
  template<> struct hash<experimental::net::v1::ip::address_v6>;

} // namespace std
\end{codeblock}



\rSec1[internet.reqmts]{Requirements}


\indextext{\idxcode{InternetProtocol}}%
\indextext{requirements!\idxcode{InternetProtocol}}%
\rSec2[internet.reqmts.protocol]{Internet protocol requirements}

\pnum
A type \tcode{X} meets the \tcode{InternetProtocol} requirements if it satisfies the requirements of \tcode{Acceptable\-Proto\-col} (\ref{socket.reqmts.acceptableprotocol}), as well as the additional requirements listed below.

\pnum
In the table below,
\tcode{a} and \tcode{b} denote values of type \tcode{X} or \tcode{const X}.

\begin{libreqtab3}
{InternetProtocol requirements}
{tab:internet.reqmts.protocol.requirements}
\\ \topline
\lhdr{expression}  &
\chdr{return type}  &
\rhdr{assertion/note pre/post-conditions}  \\ \capsep
\endfirsthead
\continuedcaption\\
\hline
\lhdr{expression}  &
\chdr{return type}  &
\rhdr{assertion/note pre/post-conditions}  \\ \capsep
\endhead

\tcode{X::resolver}  &
\tcode{ip::basic_resolver<X>}  &
 The type of a resolver for the protocol.  \\ \rowsep

\tcode{X::v4()}  &
\tcode{X}  &
 Returns an object representing the IP version 4 protocol.  \\ \rowsep

\tcode{X::v6()}  &
\tcode{X}  &
 Returns an object representing the IP version 6 protocol.  \\ \rowsep

\tcode{a == b}  &
convertible to \tcode{bool}  &
Returns \tcode{true} if \tcode{a} and \tcode{b} represent the same IP protocol version, otherwise \tcode{false}.  \\ \rowsep

\tcode{a != b}  &
convertible to \tcode{bool}  &
Returns \tcode{!(a == b)}.  \\

\end{libreqtab3}



\indextext{\idxcode{MulticastGroupSocketOption}}%
\indextext{requirements!\idxcode{MulticastGroupSocketOption}}%
\rSec2[internet.reqmts.opt.mcast]{Multicast group socket options}

\pnum
A type \tcode{X} meets the \tcode{MulticastGroupSocketOption} requirements if it satisfies the requirements of \tcode{Destruct\-ible} (\CppXref{destructible}), \tcode{CopyConstructible} (\CppXref{copyconstructible}), \tcode{CopyAssignable} (\CppXref{copyassignable}), and \tcode{SettableSocketOption}~(\ref{socket.reqmts.settablesocketoption}), as well as the additional requirements listed below.

\pnum
In the table below,
\tcode{a} denotes a value of type \tcode{address} or \tcode{const address},
\tcode{b} and \tcode{c} denote values of type \tcode{address_v4} or \tcode{const address_v4},
\tcode{d} denotes a value of type \tcode{address_v6} or \tcode{const address_v6},
\tcode{e} denotes a value of type \tcode{unsigned int} or \tcode{const unsigned int},
and \tcode{u} denotes an identifier.

\begin{libreqtab3}
{MulticastGroupSocketOption requirements}
{tab:internet.reqmts.opt.mcast.requirements}
\\ \topline
\lhdr{expression}  &
\chdr{type}  &
\rhdr{assertion/note pre/post-conditions}  \\ \capsep
\endfirsthead
\continuedcaption\\
\hline
\lhdr{expression}  &
\chdr{type}  &
\rhdr{assertion/note pre/post-conditions}  \\ \capsep
\endhead

\tcode{X u(a);}  &
  &
 Constructs a multicast group socket option to join the group with the specified version-independent address.  \\ \rowsep

\tcode{X u(b, c);}  &
  &
 Constructs a multicast group socket option to join the specified IPv4 address on a specified network interface.  \\ \rowsep

\tcode{X u(d, e);}  &
  &
 Constructs a multicast group socket option to join the specified IPv6 address on a specified network interface.  \\

\end{libreqtab3}

\pnum
In this Technical Specification, types that satisfy the \tcode{MulticastGroupSocketOption} requirements are defined as follows.

\begin{codeblock}
class @\placeholder{C}@
{
public:
  // constructors:
  explicit @\placeholder{C}@(const address& multicast_group) noexcept;
  explicit @\placeholder{C}@(const address_v4& multicast_group,
             const address_v4& network_interface = address_v4::any()) noexcept;
  explicit @\placeholder{C}@(const address_v6& multicast_group,
             unsigned int network_interface = 0) noexcept;
};
\end{codeblock}

\indexextensible{MulticastGroupSocketOption}%
\pnum
Extensible implementations provide the following member functions:

\begin{codeblock}
class @\placeholder{C}@
{
public:
  template<class Protocol> int level(const Protocol& p) const noexcept;
  template<class Protocol> int name(const Protocol& p) const noexcept;
  template<class Protocol> const void* data(const Protocol& p) const noexcept;
  template<class Protocol> size_t size(const Protocol& p) const noexcept;
  // \textit{remainder unchanged}
private:
  ip_mreq v4_value_; // \expos
  ipv6_mreq v6_value_; // \expos
};
\end{codeblock}

\pnum
Let \placeholder{L} and \placeholder{N} identify the POSIX macros to be passed as the \tcode{level} and \tcode{option_name} arguments, respectively, to POSIX \tcode{setsockopt} and \tcode{getsockopt}.

\begin{itemdecl}
explicit @\placeholder{C}@(const address& multicast_group) noexcept;
\end{itemdecl}

\begin{itemdescr}
\pnum
\effects If \tcode{multicast_group.is_v6()} is \tcode{true}, calls \placeholder{C}\tcode{(multicast_group.to_v6())}; otherwise, calls \placeholder{C}\tcode{(multicast_group.to_v4())}.
\end{itemdescr}

\begin{itemdecl}
explicit @\placeholder{C}@(const address_v4& multicast_group,
           const address_v4& network_interface = address_v4::any()) noexcept;
\end{itemdecl}

\begin{itemdescr}
\pnum
\effects For extensible implementations, \tcode{v4_value_.imr_multiaddr} is initialized to correspond to the address \tcode{multicast_group}, \tcode{v4_value_.imr_interface} is initialized to correspond to address \tcode{network_interface}, and \tcode{v6_value_} is zero-initialized.
\end{itemdescr}

\begin{itemdecl}
explicit @\placeholder{C}@(const address_v6& multicast_group,
           unsigned int network_interface = 0) noexcept;
\end{itemdecl}

\begin{itemdescr}
\pnum
\effects For extensible implementations, \tcode{v6_value_.ipv6mr_multiaddr} is initialized to correspond to the address \tcode{multicast_group}, \tcode{v6_value_.ipv6mr_interface} is initialized to \tcode{network_interface}, and \tcode{v4_value_} is zero-initialized.
\end{itemdescr}

\begin{itemdecl}
template<class Protocol> int level(const Protocol& p) const noexcept;
\end{itemdecl}

\begin{itemdescr}
\pnum
\returns \placeholder{L}.
\end{itemdescr}

\begin{itemdecl}
template<class Protocol> int name(const Protocol& p) const noexcept;
\end{itemdecl}

\begin{itemdescr}
\pnum
\returns \placeholder{N}.
\end{itemdescr}

\begin{itemdecl}
template<class Protocol> const void* data(const Protocol& p) const noexcept;
\end{itemdecl}

\begin{itemdescr}
\pnum
\returns \tcode{addressof(v6_value_)} if \tcode{p.family() == AF_INET6}, otherwise \tcode{addressof(v4_value_)}.
\end{itemdescr}

\begin{itemdecl}
template<class Protocol> size_t size(const Protocol& p) const noexcept;
\end{itemdecl}

\begin{itemdescr}
\pnum
\returns \tcode{sizeof(v6_value_)} if \tcode{p.family() == AF_INET6}, otherwise \tcode{sizeof(v4_value_)}.
\end{itemdescr}




\indextext{error codes!resolver}%
\rSec1[internet.resolver.err]{Error codes}

\indexlibrary{\idxcode{resolver_category}}%
\begin{itemdecl}
const error_category& resolver_category() noexcept;
\end{itemdecl}

\begin{itemdescr}
\pnum
\returns A reference to an object of a type derived from class \tcode{error_category}. All calls to this function return references to the same object.

\pnum
The object's \tcode{default_error_condition} and \tcode{equivalent} virtual functions behave as specified for the class \tcode{error_category}. The object's \tcode{name} virtual function returns a pointer to the string \tcode{"resolver"}.
\end{itemdescr}

\indexlibrary{\idxcode{make_error_code}}%
\indexlibrary{\idxcode{resolver_errc}!\idxcode{make_error_code}}%
\begin{itemdecl}
error_code make_error_code(resolver_errc e) noexcept;
\end{itemdecl}

\begin{itemdescr}
\pnum
\returns \tcode{error_code(static_cast<int>(e), resolver_category())}.
\end{itemdescr}

\indexlibrary{\idxcode{make_error_condition}}%
\indexlibrary{\idxcode{resolver_errc}!\idxcode{make_error_condition}}%
\begin{itemdecl}
error_condition make_error_condition(resolver_errc e) noexcept;
\end{itemdecl}

\begin{itemdescr}
\pnum
\returns \tcode{error_condition(static_cast<int>(e), resolver_category())}.
\end{itemdescr}



\rSec1[internet.address]{Class \tcode{ip::address}}

\pnum
The class \tcode{address} is a version-independent representation for an IP address. An object of class \tcode{address} holds either an IPv4 address, an IPv6 address, or no valid address.

\indexlibrary{\idxcode{address}}%
\indexlibrary{\idxcode{address}!\idxcode{is_v4}}%
\indexlibrary{\idxcode{address}!\idxcode{is_v6}}%
\indexlibrary{\idxcode{address}!\idxcode{to_v4}}%
\indexlibrary{\idxcode{address}!\idxcode{to_v6}}%
\indexlibrary{\idxcode{address}!\idxcode{is_unspecified}}%
\indexlibrary{\idxcode{address}!\idxcode{is_loopback}}%
\indexlibrary{\idxcode{address}!\idxcode{is_multicast}}%
\indexlibrary{\idxcode{address}!\idxcode{to_string}}%
\begin{codeblock}
namespace std {
namespace experimental {
namespace net {
inline namespace v1 {
namespace ip {

  class address
  {
  public:
    // \ref{internet.address.cons}, constructors:
    constexpr address() noexcept;
    constexpr address(const address& a) noexcept;
    constexpr address(const address_v4& a) noexcept;
    constexpr address(const address_v6& a) noexcept;

    // \ref{internet.address.assign}, assignment:
    address& operator=(const address& a) noexcept;
    address& operator=(const address_v4& a) noexcept;
    address& operator=(const address_v6& a) noexcept;

    // \ref{internet.address.members}, members:
    constexpr bool is_v4() const noexcept;
    constexpr bool is_v6() const noexcept;
    constexpr address_v4 to_v4() const;
    constexpr address_v6 to_v6() const;
    constexpr bool is_unspecified() const noexcept;
    constexpr bool is_loopback() const noexcept;
    constexpr bool is_multicast() const noexcept;
    template<class Allocator = allocator<char>>
      basic_string<char, char_traits<char>, Allocator>
        to_string(const Allocator& a = Allocator()) const;

  private:
    address_v4 v4_; // \expos
    address_v6 v6_; // \expos
  };

  // \ref{internet.address.comparisons}, address comparisons:
  constexpr bool operator==(const address& a, const address& b) noexcept;
  constexpr bool operator!=(const address& a, const address& b) noexcept;
  constexpr bool operator< (const address& a, const address& b) noexcept;
  constexpr bool operator> (const address& a, const address& b) noexcept;
  constexpr bool operator<=(const address& a, const address& b) noexcept;
  constexpr bool operator>=(const address& a, const address& b) noexcept;

  // \ref{internet.address.creation}, address creation:
  address make_address(const char* str);
  address make_address(const char* str, error_code& ec) noexcept;
  address make_address(const string& str);
  address make_address(const string& str, error_code& ec) noexcept;
  address make_address(string_view str);
  address make_address(string_view str, error_code& ec) noexcept;

  // \ref{internet.address.io}, address I/O:
  template<class CharT, class Traits>
    basic_ostream<CharT, Traits>& operator<<(
      basic_ostream<CharT, Traits>& os, const address& addr);

} // namespace ip
} // inline namespace v1
} // namespace net
} // namespace experimental
} // namespace std
\end{codeblock}

\indexlibrary{\idxcode{address}}%
\pnum
\tcode{address} satisfies the requirements for \tcode{Destructible} (\CppXref{destructible}), \tcode{CopyConstructible} (\CppXref{copyconstructible}), and \tcode{CopyAssignable} (\CppXref{copyassignable}).


\indexlibrary{\idxcode{address}!constructor}%
\rSec2[internet.address.cons]{\tcode{ip::address} constructors}

\begin{itemdecl}
constexpr address() noexcept;
\end{itemdecl}

\begin{itemdescr}
\pnum
\postconditions \tcode{is_v4() == true}, \tcode{is_v6() == false}, and \tcode{is_unspecified() == true}.
\end{itemdescr}

\begin{itemdecl}
constexpr address(const address_v4& a) noexcept;
\end{itemdecl}

\begin{itemdescr}
\pnum
\effects Initializes \tcode{v4_} with \tcode{a}.

\pnum
\postconditions \tcode{is_v4() == true} and \tcode{is_v6() == false}.
\end{itemdescr}

\begin{itemdecl}
constexpr address(const address_v6& a) noexcept;
\end{itemdecl}

\begin{itemdescr}
\pnum
\effects Initializes \tcode{v6_} with \tcode{a}.

\pnum
\postconditions \tcode{is_v4() == false} and \tcode{is_v6() == true}.
\end{itemdescr}



\indexlibrarymember{operator=}{address}%
\rSec2[internet.address.assign]{\tcode{ip::address} assignment}

\begin{itemdecl}
address& operator=(const address_v4& a) noexcept;
\end{itemdecl}

\begin{itemdescr}
\pnum
\postconditions \tcode{is_v4() == true} and \tcode{is_v6() == false} and \tcode{to_v4() == a}.

\pnum
\returns \tcode{*this}
\end{itemdescr}

\begin{itemdecl}
address& operator=(const address_v6& a) noexcept;
\end{itemdecl}

\begin{itemdescr}
\pnum
\postconditions \tcode{is_v4() == false} and \tcode{is_v6() == true} and \tcode{to_v6() == a}.

\pnum
\returns \tcode{*this}
\end{itemdescr}



\rSec2[internet.address.members]{\tcode{ip::address} members}

\indexlibrarymember{is_v4}{address}%
\begin{itemdecl}
constexpr bool is_v4() const noexcept;
\end{itemdecl}

\begin{itemdescr}
\pnum
\returns \tcode{true} if the object contains an IP version 4 address, otherwise \tcode{false}.
\end{itemdescr}

\indexlibrarymember{is_v6}{address}%
\begin{itemdecl}
constexpr bool is_v6() const noexcept;
\end{itemdecl}

\begin{itemdescr}
\pnum
\returns \tcode{true} if the object contains an IP version 6 address, otherwise \tcode{false}.
\end{itemdescr}

\indexlibrarymember{to_v4}{address}%
\begin{itemdecl}
constexpr address_v4 to_v4() const;
\end{itemdecl}

\begin{itemdescr}
\pnum
\returns \tcode{v4_}.

\pnum
\remarks \tcode{bad_address_cast} if \tcode{is_v4() == false}.
\end{itemdescr}

\indexlibrarymember{to_v6}{address}%
\begin{itemdecl}
constexpr address_v6 to_v6() const;
\end{itemdecl}

\begin{itemdescr}
\pnum
\returns \tcode{v6_}.

\pnum
\remarks \tcode{bad_address_cast} if \tcode{is_v6() == false}.
\end{itemdescr}

\indexlibrarymember{is_unspecified}{address}%
\begin{itemdecl}
constexpr bool is_unspecified() const noexcept;
\end{itemdecl}

\begin{itemdescr}
\pnum
\returns If \tcode{is_v4()}, returns \tcode{v4_.is_unspecified()}. Otherwise returns \tcode{v6_.is_unspecified()}.
\end{itemdescr}

\indexlibrarymember{is_loopback}{address}%
\begin{itemdecl}
constexpr bool is_loopback() const noexcept;
\end{itemdecl}

\begin{itemdescr}
\pnum
\returns If \tcode{is_v4()}, returns \tcode{v4_.is_loopback()}. Otherwise returns \tcode{v6_.is_loopback()}.
\end{itemdescr}

\indexlibrarymember{is_multicast}{address}%
\begin{itemdecl}
constexpr bool is_multicast() const noexcept;
\end{itemdecl}

\begin{itemdescr}
\pnum
\returns If \tcode{is_v4()}, returns \tcode{v4_.is_multicast()}. Otherwise returns \tcode{v6_.is_multicast()}.
\end{itemdescr}

\indexlibrarymember{to_string}{address}%
\begin{itemdecl}
template<class Allocator = allocator<char>>
  basic_string<char, char_traits<char>, Allocator>
    to_string(const Allocator& a = Allocator()) const;
\end{itemdecl}

\begin{itemdescr}
\pnum
\returns If \tcode{is_v4()}, returns \tcode{v4_.to_string(a)}. Otherwise returns \tcode{v6_.to_string(a)}.
\end{itemdescr}


\indexlibrarymember{operator==}{address}%
\rSec2[internet.address.comparisons]{\tcode{ip::address} comparisons}

\begin{itemdecl}
constexpr bool operator==(const address& a, const address& b) noexcept;
\end{itemdecl}

\begin{itemdescr}
\pnum
\returns
\begin{itemize}
\item
 if \tcode{a.is_v4() != b.is_v4()}, \tcode{false};
\item
 if \tcode{a.is_v4()}, the result of \tcode{a.v4_ == b.v4_};
\item
 otherwise, the result of \tcode{a.v6_ == b.v6_}.
\end{itemize}
\end{itemdescr}

\indexlibrarymember{operator"!=}{address}%
\begin{itemdecl}
constexpr bool operator!=(const address& a, const address& b) noexcept;
\end{itemdecl}

\begin{itemdescr}
\pnum
\returns \tcode{!(a == b)}.
\end{itemdescr}

\indexlibrarymember{operator<}{address}%
\begin{itemdecl}
constexpr bool operator< (const address& a, const address& b) noexcept;
\end{itemdecl}

\begin{itemdescr}
\pnum
\returns
\begin{itemize}
\item
 if \tcode{a.is_v4() \&\& !b.is_v4()}, \tcode{true};
\item
 if \tcode{!a.is_v4() \&\& b.is_v4()}, \tcode{false};
\item
 if \tcode{a.is_v4()}, the result of \tcode{a.v4_ < b.v4_};
\item
 otherwise, the result of \tcode{a.v6_ < b.v6_}.
\end{itemize}
\end{itemdescr}

\indexlibrarymember{operator>}{address}%
\begin{itemdecl}
constexpr bool operator> (const address& a, const address& b) noexcept;
\end{itemdecl}

\begin{itemdescr}
\pnum
\returns \tcode{b < a}.
\end{itemdescr}

\indexlibrarymember{operator<=}{address}%
\begin{itemdecl}
constexpr bool operator<=(const address& a, const address& b) noexcept;
\end{itemdecl}

\begin{itemdescr}
\pnum
\returns \tcode{!(b < a)}.
\end{itemdescr}

\indexlibrarymember{operator>=}{address}%
\begin{itemdecl}
constexpr bool operator>=(const address& a, const address& b) noexcept;
\end{itemdecl}

\begin{itemdescr}
\pnum
\returns \tcode{!(a < b)}.
\end{itemdescr}



\rSec2[internet.address.creation]{\tcode{ip::address} creation}

\indexlibrary{\idxcode{make_address}}%
\begin{itemdecl}
address make_address(const char* str);
address make_address(const char* str, error_code& ec) noexcept;
address make_address(const string& str);
address make_address(const string& str, error_code& ec) noexcept;
address make_address(string_view str);
address make_address(string_view str, error_code& ec) noexcept;
\end{itemdecl}

\begin{itemdescr}
\pnum
\effects Converts a textual representation of an address into an object of class \tcode{address}, as if by calling:
\begin{codeblock}
address a;
address_v6 v6a = make_address_v6(str, ec);
if (!ec)
  a = v6a;
else
{
  address_v4 v4a = make_address_v4(str, ec);
  if (!ec)
    a = v4a;
}
\end{codeblock}


\pnum
\returns \tcode{a}.
\end{itemdescr}



\rSec2[internet.address.io]{\tcode{ip::address} I/O}

\indexlibrarymember{operator<<}{address}%
\begin{itemdecl}
template<class CharT, class Traits>
  basic_ostream<CharT, Traits>& operator<<(
    basic_ostream<CharT, Traits>& os, const address& addr);
\end{itemdecl}

\begin{itemdescr}
\pnum
\returns \tcode{os << addr.to_string().c_str()}.
\end{itemdescr}




\rSec1[internet.address.v4]{Class \tcode{ip::address_v4}}

\pnum
The class \tcode{address_v4} is a representation of an IPv4 address.

\indexlibrary{\idxcode{address_v4}}%
\begin{codeblock}
namespace std {
namespace experimental {
namespace net {
inline namespace v1 {
namespace ip {

  class address_v4
  {
  public:
    // \ref{internet.address.v4.bytes}, types:
    using uint_type = uint_least32_t;
    struct bytes_type;

    // \ref{internet.address.v4.cons}, constructors:
    constexpr address_v4() noexcept;
    constexpr address_v4(const address_v4& a) noexcept;
    constexpr address_v4(const bytes_type& bytes);
    explicit constexpr address_v4(uint_type val);

    // assignment:
    address_v4& operator=(const address_v4& a) noexcept;

    // \ref{internet.address.v4.members}, members:
    constexpr bool is_unspecified() const noexcept;
    constexpr bool is_loopback() const noexcept;
    constexpr bool is_multicast() const noexcept;
    constexpr bytes_type to_bytes() const noexcept;
    constexpr uint_type to_uint() const noexcept;
    template<class Allocator = allocator<char>>
      basic_string<char, char_traits<char>, Allocator>
        to_string(const Allocator& a = Allocator()) const;

    // \ref{internet.address.v4.static}, static members:
    static constexpr address_v4 any() noexcept;
    static constexpr address_v4 loopback() noexcept;
    static constexpr address_v4 broadcast() noexcept;
  };

  // \ref{internet.address.v4.comparisons}, address_v4 comparisons:
  constexpr bool operator==(const address_v4& a, const address_v4& b) noexcept;
  constexpr bool operator!=(const address_v4& a, const address_v4& b) noexcept;
  constexpr bool operator< (const address_v4& a, const address_v4& b) noexcept;
  constexpr bool operator> (const address_v4& a, const address_v4& b) noexcept;
  constexpr bool operator<=(const address_v4& a, const address_v4& b) noexcept;
  constexpr bool operator>=(const address_v4& a, const address_v4& b) noexcept;

  // \ref{internet.address.v4.creation}, address_v4 creation:
  constexpr address_v4 make_address_v4(const address_v4::bytes_type& bytes);
  constexpr address_v4 make_address_v4(address_v4::uint_type val);
  constexpr address_v4 make_address_v4(v4_mapped_t, const address_v6& a);
  address_v4 make_address_v4(const char* str);
  address_v4 make_address_v4(const char* str, error_code& ec) noexcept;
  address_v4 make_address_v4(const string& str);
  address_v4 make_address_v4(const string& str, error_code& ec) noexcept;
  address_v4 make_address_v4(string_view str);
  address_v4 make_address_v4(string_view str, error_code& ec) noexcept;

  // \ref{internet.address.v4.io}, address_v4 I/O:
  template<class CharT, class Traits>
    basic_ostream<CharT, Traits>& operator<<(
      basic_ostream<CharT, Traits>& os, const address_v4& addr);

} // namespace ip
} // inline namespace v1
} // namespace net
} // namespace experimental
} // namespace std
\end{codeblock}

\pnum
\tcode{address_v4} satisfies the requirements for \tcode{Destructible} (\CppXref{destructible}), \tcode{CopyConstructible} (\CppXref{copyconstructible}), and \tcode{CopyAssignable} (\CppXref{copyassignable}).


\rSec2[internet.address.v4.bytes]{Struct \tcode{ip::address_v4::bytes_type}}

\indexlibrarymember{bytes_type}{address_v4}%
\begin{codeblock}
namespace std {
namespace experimental {
namespace net {
inline namespace v1 {
namespace ip {

  struct address_v4::bytes_type : array<unsigned char, 4>
  {
    template<class... T> explicit constexpr bytes_type(T... t)
      : array<unsigned char, 4>{{static_cast<unsigned char>(t)...}} {}
  };

} // namespace ip
} // inline namespace v1
} // namespace net
} // namespace experimental
} // namespace std
\end{codeblock}

\pnum
The \tcode{ip::address_v4::bytes_type} type is a standard-layout struct that provides a byte-level representation of an IPv4 address in network byte order.



\indexlibrary{\idxcode{address_v4}!constructor}%
\rSec2[internet.address.v4.cons]{\tcode{ip::address_v4} constructors}

\begin{itemdecl}
constexpr address_v4() noexcept;
\end{itemdecl}

\begin{itemdescr}
\pnum
\postconditions \tcode{to_bytes()} yields \tcode{\{0, 0, 0, 0\}} and \tcode{to_uint() == 0}.
\end{itemdescr}

\begin{itemdecl}
constexpr address_v4(const bytes_type& bytes);
\end{itemdecl}

\begin{itemdescr}
\pnum
\remarks \tcode{out_of_range} if any element of \tcode{bytes} is not in the range \tcode{[0, 0xFF]}. \begin{note} For implementations where \tcode{numeric_limits<unsigned char>::max() == 0xFF}, no out-of-range detection is needed. \end{note}

\pnum
\postconditions \tcode{to_bytes() == bytes} and \tcode{to_uint() == (bytes[0] << 24) | (bytes[1] << 16) | (bytes[2] << 8) | bytes[3]}.
\end{itemdescr}

\begin{itemdecl}
explicit constexpr address_v4(address_v4::uint_type val);
\end{itemdecl}

\begin{itemdescr}
\pnum
\remarks \tcode{out_of_range} if \tcode{val} is not in the range \tcode{[0, 0xFFFFFFFF]}. \begin{note} For implementations where \tcode{numeric_limits<address_v4::uint_type>::max() == 0xFFFFFFFF}, no out-of-range detection is needed. \end{note}

\pnum
\postconditions \tcode{to_uint() == val} and \tcode{to_bytes()} is:
\begin{codeblock}
{ (val >> 24) & 0xFF, (val >> 16) & 0xFF, (val >> 8) & 0xFF, val & 0xFF }
\end{codeblock}
\end{itemdescr}



\rSec2[internet.address.v4.members]{\tcode{ip::address_v4} members}

\indexlibrarymember{is_unspecified}{address_v4}%
\begin{itemdecl}
constexpr bool is_unspecified() const noexcept;
\end{itemdecl}

\begin{itemdescr}
\pnum
\returns \tcode{to_uint() == 0}.
\end{itemdescr}

\indexlibrarymember{is_loopback}{address_v4}%
\begin{itemdecl}
constexpr bool is_loopback() const noexcept;
\end{itemdecl}

\begin{itemdescr}
\pnum
\returns \tcode{(to_uint() \& 0xFF000000) == 0x7F000000}.
\end{itemdescr}

\indexlibrarymember{is_multicast}{address_v4}%
\begin{itemdecl}
constexpr bool is_multicast() const noexcept;
\end{itemdecl}

\begin{itemdescr}
\pnum
\returns \tcode{(to_uint() \& 0xF0000000) == 0xE0000000}.
\end{itemdescr}

\indexlibrarymember{to_bytes}{address_v4}%
\begin{itemdecl}
constexpr bytes_type to_bytes() const noexcept;
\end{itemdecl}

\begin{itemdescr}
\pnum
\returns A representation of the address in network byte order~(\ref{defs.net.byte.order}).
\end{itemdescr}

\indexlibrarymember{to_uint}{address_v4}%
\begin{itemdecl}
constexpr address_v4::uint_type to_uint() const noexcept;
\end{itemdecl}

\begin{itemdescr}
\pnum
\returns A representation of the address in host byte order~(\ref{defs.host.byte.order}).
\end{itemdescr}

\indexlibrarymember{to_string}{address_v4}%
\begin{itemdecl}
template<class Allocator = allocator<char>>
  basic_string<char, char_traits<char>, Allocator>
    to_string(const Allocator& a = Allocator()) const;
\end{itemdecl}

\begin{itemdescr}
\pnum
\returns If successful, the textual representation of the address, determined as if by POSIX \tcode{inet_ntop} when invoked with address family \tcode{AF_INET}. Otherwise \tcode{basic_string<char, char_traits<char>, Allocator>(a)}.
\end{itemdescr}



\rSec2[internet.address.v4.static]{\tcode{ip::address_v4} static members}

\indexlibrarymember{any}{address_v4}%
\begin{itemdecl}
static constexpr address_v4 any() noexcept;
\end{itemdecl}

\begin{itemdescr}
\pnum
\returns \tcode{address_v4()}.
\end{itemdescr}

\indexlibrarymember{loopback}{address_v4}%
\begin{itemdecl}
static constexpr address_v4 loopback() noexcept;
\end{itemdecl}

\begin{itemdescr}
\pnum
\returns \tcode{address_v4(0x7F000001)}.
\end{itemdescr}

\indexlibrarymember{multicast}{address_v4}%
\begin{itemdecl}
static constexpr address_v4 broadcast() noexcept;
\end{itemdecl}

\begin{itemdescr}
\pnum
\returns \tcode{address_v4(0xFFFFFFFF)}.
\end{itemdescr}



\rSec2[internet.address.v4.comparisons]{\tcode{ip::address_v4} comparisons}

\indexlibrarymember{operator==}{address_v4}%
\begin{itemdecl}
constexpr bool operator==(const address_v4& a, const address_v4& b) noexcept;
\end{itemdecl}

\begin{itemdescr}
\pnum
\returns \tcode{a.to_uint() == b.to_uint()}.
\end{itemdescr}

\indexlibrarymember{operator"!=}{address_v4}%
\begin{itemdecl}
constexpr bool operator!=(const address_v4& a, const address_v4& b) noexcept;
\end{itemdecl}

\begin{itemdescr}
\pnum
\returns \tcode{!(a == b)}.
\end{itemdescr}

\indexlibrarymember{operator<}{address_v4}%
\begin{itemdecl}
constexpr bool operator< (const address_v4& a, const address_v4& b) noexcept;
\end{itemdecl}

\begin{itemdescr}
\pnum
\returns \tcode{a.to_uint() < b.to_uint()}.
\end{itemdescr}

\indexlibrarymember{operator>}{address_v4}%
\begin{itemdecl}
constexpr bool operator> (const address_v4& a, const address_v4& b) noexcept;
\end{itemdecl}

\begin{itemdescr}
\pnum
\returns \tcode{b < a}.
\end{itemdescr}

\indexlibrarymember{operator<=}{address_v4}%
\begin{itemdecl}
constexpr bool operator<=(const address_v4& a, const address_v4& b) noexcept;
\end{itemdecl}

\begin{itemdescr}
\pnum
\returns \tcode{!(b < a)}.
\end{itemdescr}

\indexlibrarymember{operator>=}{address_v4}%
\begin{itemdecl}
constexpr bool operator>=(const address_v4& a, const address_v4& b) noexcept;
\end{itemdecl}

\begin{itemdescr}
\pnum
\returns \tcode{!(a < b)}.
\end{itemdescr}



\indexlibrary{\idxcode{make_address_v4}}%
\rSec2[internet.address.v4.creation]{\tcode{ip::address_v4} creation}

\begin{itemdecl}
constexpr address_v4 make_address_v4(const address_v4::bytes_type& bytes);
\end{itemdecl}

\begin{itemdescr}
\pnum
\returns \tcode{address_v4(bytes)}.
\end{itemdescr}

\begin{itemdecl}
constexpr address_v4 make_address_v4(address_v4::uint_type val);
\end{itemdecl}

\begin{itemdescr}
\pnum
\returns \tcode{address_v4(val)}.
\end{itemdescr}

\begin{itemdecl}
constexpr address_v4 make_address_v4(v4_mapped_t, const address_v6& a);
\end{itemdecl}

\begin{itemdescr}
\pnum
\returns An \tcode{address_v4} object corresponding to the IPv4-mapped IPv6 address, as if computed by the following method:
\begin{codeblock}
address_v6::bytes_type v6b = a.to_bytes();
address_v4::bytes_type v4b(v6b[12], v6b[13], v6b[14], v6b[15]);
return address_v4(v4b);
\end{codeblock}


\pnum
\remarks \tcode{bad_address_cast} if \tcode{a.is_v4_mapped()} is \tcode{false}.
\end{itemdescr}

\begin{itemdecl}
address_v4 make_address_v4(const char* str);
address_v4 make_address_v4(const char* str, error_code& ec) noexcept;
address_v4 make_address_v4(const string& str);
address_v4 make_address_v4(const string& str, error_code& ec) noexcept;
address_v4 make_address_v4(string_view str);
address_v4 make_address_v4(string_view str, error_code& ec) noexcept;
\end{itemdecl}

\begin{itemdescr}
\pnum
\effects Converts a textual representation of an address into a corresponding \tcode{address_v4} value, as if by POSIX \tcode{inet_pton} when invoked with address family \tcode{AF_INET}.

\pnum
\returns If successful, an \tcode{address_v4} value corresponding to the string \tcode{str}. Otherwise \tcode{address_v4()}.

\pnum
\errors
\begin{itemize}
\item
\tcode{errc::invalid_argument} --- if \tcode{str} is not a valid textual representation of an IPv4 address.
\end{itemize}
\end{itemdescr}



\rSec2[internet.address.v4.io]{\tcode{ip::address_v4} I/O}

\indexlibrarymember{operator<<}{address_v4}%
\begin{itemdecl}
template<class CharT, class Traits>
  basic_ostream<CharT, Traits>& operator<<(
    basic_ostream<CharT, Traits>& os, const address_v4& addr);
\end{itemdecl}

\begin{itemdescr}
\pnum
\returns \tcode{os << addr.to_string().c_str()}.
\end{itemdescr}




\rSec1[internet.address.v6]{Class \tcode{ip::address_v6}}

\pnum
The class \tcode{address_v6} is a representation of an IPv6 address.

\indexlibrary{\idxcode{address_v6}}%
\begin{codeblock}
namespace std {
namespace experimental {
namespace net {
inline namespace v1 {
namespace ip {

  class address_v6
  {
  public:
    // \ref{internet.address.v6.bytes}, types:
    struct bytes_type;

    // \ref{internet.address.v6.cons}, constructors:
    constexpr address_v6() noexcept;
    constexpr address_v6(const address_v6& a) noexcept;
    constexpr address_v6(const bytes_type& bytes,
                         scope_id_type scope = 0);

    // assignment:
    address_v6& operator=(const address_v6& a) noexcept;

    // \ref{internet.address.v6.members}, members:
    void scope_id(scope_id_type id) noexcept;
    constexpr scope_id_type scope_id() const noexcept;
    constexpr bool is_unspecified() const noexcept;
    constexpr bool is_loopback() const noexcept;
    constexpr bool is_multicast() const noexcept;
    constexpr bool is_link_local() const noexcept;
    constexpr bool is_site_local() const noexcept;
    constexpr bool is_v4_mapped() const noexcept;
    constexpr bool is_multicast_node_local() const noexcept;
    constexpr bool is_multicast_link_local() const noexcept;
    constexpr bool is_multicast_site_local() const noexcept;
    constexpr bool is_multicast_org_local() const noexcept;
    constexpr bool is_multicast_global() const noexcept;
    constexpr bytes_type to_bytes() const noexcept;
    template<class Allocator = allocator<char>>
      basic_string<char, char_traits<char>, Allocator>
        to_string(const Allocator& a = Allocator()) const;

    // \ref{internet.address.v6.static}, static members:
    static constexpr address_v6 any() noexcept;
    static constexpr address_v6 loopback() noexcept;
  };

  // \ref{internet.address.v6.comparisons}, address_v6 comparisons:
  constexpr bool operator==(const address_v6& a, const address_v6& b) noexcept;
  constexpr bool operator!=(const address_v6& a, const address_v6& b) noexcept;
  constexpr bool operator< (const address_v6& a, const address_v6& b) noexcept;
  constexpr bool operator> (const address_v6& a, const address_v6& b) noexcept;
  constexpr bool operator<=(const address_v6& a, const address_v6& b) noexcept;
  constexpr bool operator>=(const address_v6& a, const address_v6& b) noexcept;

  // \ref{internet.address.v6.creation}, address_v6 creation:
  constexpr address_v6 make_address_v6(const address_v6::bytes_type& bytes,
                                       scope_id_type scope_id = 0);
  constexpr address_v6 make_address_v6(v4_mapped_t, const address_v4& a) noexcept;
  address_v6 make_address_v6(const char* str);
  address_v6 make_address_v6(const char* str, error_code& ec) noexcept;
  address_v6 make_address_v6(const string& str);
  address_v6 make_address_v6(const string& str, error_code& ec) noexcept;
  address_v6 make_address_v6(string_view str);
  address_v6 make_address_v6(string_view str, error_code& ec) noexcept;

  // \ref{internet.address.v6.io}, address_v6 I/O:
  template<class CharT, class Traits>
    basic_ostream<CharT, Traits>& operator<<(
      basic_ostream<CharT, Traits>& os, const address_v6& addr);

} // namespace ip
} // inline namespace v1
} // namespace net
} // namespace experimental
} // namespace std
\end{codeblock}

\pnum
\tcode{address_v6} satisfies the requirements for \tcode{Destructible} (\CppXref{destructible}), \tcode{CopyConstructible} (\CppXref{copyconstructible}), and \tcode{CopyAssignable} (\CppXref{copyassignable}).

\pnum
\begin{note} The implementations of the functions \tcode{is_unspecified}, \tcode{is_loopback}, \tcode{is_multicast}, \tcode{is_link_local}, \tcode{is_site_local}, \tcode{is_v4_mapped}, \tcode{is_multicast_node_local}, \tcode{is_multicast_link_local}, \tcode{is_mul\-ti\-cast_site_local}, \tcode{is_multicast_org_local} and \tcode{is_multicast_global} are determined by [RFC4291]. \end{note}


\rSec2[internet.address.v6.bytes]{Struct \tcode{ip::address_v6::bytes_type}}

\indexlibrarymember{bytes_type}{address_v6}%
\begin{codeblock}
namespace std {
namespace experimental {
namespace net {
inline namespace v1 {
namespace ip {

  struct address_v6::bytes_type : array<unsigned char, 16>
  {
    template<class... T> explicit constexpr bytes_type(T... t)
      : array<unsigned char, 16>{{static_cast<unsigned char>(t)...}} {}
  };

} // namespace ip
} // inline namespace v1
} // namespace net
} // namespace experimental
} // namespace std
\end{codeblock}

\pnum
The \tcode{ip::address_v6::bytes_type} type is a standard-layout struct that provides a byte-level representation of an IPv6 address in network byte order.



\indexlibrary{\idxcode{address_v6}!constructor}%
\rSec2[internet.address.v6.cons]{\tcode{ip::address_v6} constructors}

\begin{itemdecl}
constexpr address_v6() noexcept;
\end{itemdecl}

\begin{itemdescr}
\pnum
\postconditions \tcode{is_unspecified() == true} and \tcode{scope_id() == 0}.
\end{itemdescr}

\begin{itemdecl}
constexpr address_v6(const bytes_type& bytes,
                     scope_id_type scope = 0);
\end{itemdecl}

\begin{itemdescr}
\pnum
\remarks \tcode{out_of_range} if any element of \tcode{bytes} is not in the range \crange{0}{0xFF}. \begin{note} For implementations where \tcode{numeric_limits<unsigned char>::max() == 0xFF}, no out-of-range detection is needed. \end{note}

\pnum
\postconditions \tcode{to_bytes() == bytes} and \tcode{scope_id() == scope}.
\end{itemdescr}



\rSec2[internet.address.v6.members]{\tcode{ip::address_v6} members}

\indexlibrarymember{scope_id}{address_v6}%
\begin{itemdecl}
void scope_id(scope_id_type id) noexcept;
\end{itemdecl}

\begin{itemdescr}
\pnum
\postconditions \tcode{scope_id() == id}.
\end{itemdescr}

\begin{itemdecl}
constexpr scope_id_type scope_id() const noexcept;
\end{itemdecl}

\begin{itemdescr}
\pnum
\returns The scope identifier associated with the address.
\end{itemdescr}

\indexlibrarymember{is_unspecified}{address_v6}%
\begin{itemdecl}
constexpr bool is_unspecified() const noexcept;
\end{itemdecl}

\begin{itemdescr}
\pnum
\returns \tcode{*this == make_address_v6("::")}.
\end{itemdescr}

\indexlibrarymember{is_loopback}{address_v6}%
\begin{itemdecl}
constexpr bool is_loopback() const noexcept;
\end{itemdecl}

\begin{itemdescr}
\pnum
\returns \tcode{*this == make_address_v6("::1")}.
\end{itemdescr}

\indexlibrarymember{is_multicast}{address_v6}%
\begin{itemdecl}
constexpr bool is_multicast() const noexcept;
\end{itemdecl}

\begin{itemdescr}
\pnum
\returns A boolean indicating whether the \tcode{address_v6} object represents a multicast address, as if computed by the following method:
\begin{codeblock}
bytes_type b = to_bytes();
return b[0] == 0xFF;
\end{codeblock}

\end{itemdescr}

\indexlibrarymember{is_link_local}{address_v6}%
\begin{itemdecl}
constexpr bool is_link_local() const noexcept;
\end{itemdecl}

\begin{itemdescr}
\pnum
\returns A boolean indicating whether the \tcode{address_v6} object represents a unicast link-local address, as if computed by the following method:
\begin{codeblock}
bytes_type b = to_bytes();
return b[0] == 0xFE && (b[1] & 0xC0) == 0x80;
\end{codeblock}

\end{itemdescr}

\indexlibrarymember{is_site_local}{address_v6}%
\begin{itemdecl}
constexpr bool is_site_local() const noexcept;
\end{itemdecl}

\begin{itemdescr}
\pnum
\returns A boolean indicating whether the \tcode{address_v6} object represents a unicast site-local address, as if computed by the following method:
\begin{codeblock}
bytes_type b = to_bytes();
return b[0] == 0xFE && (b[1] & 0xC0) == 0xC0;
\end{codeblock}

\end{itemdescr}

\indexlibrarymember{is_v4_mapped}{address_v6}%
\begin{itemdecl}
constexpr bool is_v4_mapped() const noexcept;
\end{itemdecl}

\begin{itemdescr}
\pnum
\returns A boolean indicating whether the \tcode{address_v6} object represents an IPv4-mapped IPv6 address, as if computed by the following method:
\begin{codeblock}
bytes_type b = to_bytes();
return b[ 0] == 0 && b[ 1] == 0 && b[ 2] == 0    && b[ 3] == 0
    && b[ 4] == 0 && b[ 5] == 0 && b[ 6] == 0    && b[ 7] == 0
    && b[ 8] == 0 && b[ 9] == 0 && b[10] == 0xFF && b[11] == 0xFF;
\end{codeblock}

\end{itemdescr}

\indexlibrarymember{is_multicast_node_local}{address_v6}%
\begin{itemdecl}
constexpr bool is_multicast_node_local() const noexcept;
\end{itemdecl}

\begin{itemdescr}
\pnum
\returns \tcode{is_multicast() \&\& (to_bytes()[1] \& 0x0F) == 0x01}.
\end{itemdescr}

\indexlibrarymember{is_multicast_link_local}{address_v6}%
\begin{itemdecl}
constexpr bool is_multicast_link_local() const noexcept;
\end{itemdecl}

\begin{itemdescr}
\pnum
\returns \tcode{is_multicast() \&\& (to_bytes()[1] \& 0x0F) == 0x02}.
\end{itemdescr}

\indexlibrarymember{is_multicast_site_local}{address_v6}%
\begin{itemdecl}
constexpr bool is_multicast_site_local() const noexcept;
\end{itemdecl}

\begin{itemdescr}
\pnum
\returns \tcode{is_multicast() \&\& (to_bytes()[1] \& 0x0F) == 0x05}.
\end{itemdescr}

\indexlibrarymember{is_multicast_org_local}{address_v6}%
\begin{itemdecl}
constexpr bool is_multicast_org_local() const noexcept;
\end{itemdecl}

\begin{itemdescr}
\pnum
\returns \tcode{is_multicast() \&\& (to_bytes()[1] \& 0x0F) == 0x08}.
\end{itemdescr}

\indexlibrarymember{is_multicast_global}{address_v6}%
\begin{itemdecl}
constexpr bool is_multicast_global() const noexcept;
\end{itemdecl}

\begin{itemdescr}
\pnum
\returns \tcode{is_multicast() \&\& (to_bytes()[1] \& 0x0F) == 0x0E}.
\end{itemdescr}

\indexlibrarymember{to_bytes}{address_v6}%
\begin{itemdecl}
constexpr bytes_type to_bytes() const noexcept;
\end{itemdecl}

\begin{itemdescr}
\pnum
\returns A representation of the address in network byte order~(\ref{defs.net.byte.order}).
\end{itemdescr}

\indexlibrarymember{to_string}{address_v6}%
\begin{itemdecl}
template<class Allocator = allocator<char>>
  basic_string<char, char_traits<char>, Allocator>
    to_string(const Allocator& a = Allocator()) const;
\end{itemdecl}

\begin{itemdescr}
\pnum
\effects Converts an address into a textual representation. If \tcode{scope_id() == 0}, converts as if by POSIX \tcode{inet_ntop} when invoked with address family \tcode{AF_INET6}. If \tcode{scope_id() != 0}, the format is \tcode{address}\tcode{\%}\tcode{scope-id}, where \tcode{address} is the textual representation of the equivalent address having \tcode{scope_id() == 0}, and \tcode{scope-id} is an \impldef{textual representation of IPv6 scope identifiers} textual representation of the scope identifier.

\pnum
\returns If successful, the textual representation of the address. Otherwise \tcode{basic_string<char, char_traits<char>, Allocator>(a)}.
\end{itemdescr}



\rSec2[internet.address.v6.static]{\tcode{ip::address_v6} static members}

\indexlibrarymember{any}{address_v6}%
\begin{itemdecl}
static constexpr address_v6 any() noexcept;
\end{itemdecl}

\begin{itemdescr}
\pnum
\returns An address \tcode{a} such that the \tcode{a.is_unspecified() == true} and \tcode{a.scope_id() == 0}.
\end{itemdescr}

\indexlibrarymember{loopback}{address_v6}%
\begin{itemdecl}
static constexpr address_v6 loopback() noexcept;
\end{itemdecl}

\begin{itemdescr}
\pnum
\returns An address \tcode{a} such that the \tcode{a.is_loopback() == true} and \tcode{a.scope_id() == 0}.
\end{itemdescr}



\rSec2[internet.address.v6.comparisons]{\tcode{ip::address_v6} comparisons}

\indexlibrarymember{operator==}{address_v6}%
\begin{itemdecl}
constexpr bool operator==(const address_v6& a, const address_v6& b) noexcept;
\end{itemdecl}

\begin{itemdescr}
\pnum
\returns \tcode{a.to_bytes() == b.to_bytes() \&\& a.scope_id() == b.scope_id()}.
\end{itemdescr}

\indexlibrarymember{operator"!=}{address_v6}%
\begin{itemdecl}
constexpr bool operator!=(const address_v6& a, const address_v6& b) noexcept;
\end{itemdecl}

\begin{itemdescr}
\pnum
\returns \tcode{!(a == b)}.
\end{itemdescr}

\indexlibrarymember{operator<}{address_v6}%
\begin{itemdecl}
constexpr bool operator< (const address_v6& a, const address_v6& b) noexcept;
\end{itemdecl}

\begin{itemdescr}
\pnum
\returns \tcode{tie(a.to_bytes(), a.scope_id()) < tie(b.to_bytes(), b.scope_id())}.
\end{itemdescr}

\indexlibrarymember{operator>}{address_v6}%
\begin{itemdecl}
constexpr bool operator> (const address_v6& a, const address_v6& b) noexcept;
\end{itemdecl}

\begin{itemdescr}
\pnum
\returns \tcode{b < a}.
\end{itemdescr}

\indexlibrarymember{operator<=}{address_v6}%
\begin{itemdecl}
constexpr bool operator<=(const address_v6& a, const address_v6& b) noexcept;
\end{itemdecl}

\begin{itemdescr}
\pnum
\returns \tcode{!(b < a)}.
\end{itemdescr}

\indexlibrarymember{operator>=}{address_v6}%
\begin{itemdecl}
constexpr bool operator>=(const address_v6& a, const address_v6& b) noexcept;
\end{itemdecl}

\begin{itemdescr}
\pnum
\returns \tcode{!(a < b)}.
\end{itemdescr}



\indexlibrary{\idxcode{make_address_v6}}%
\rSec2[internet.address.v6.creation]{\tcode{ip::address_v6} creation}

\begin{itemdecl}
constexpr address_v6 make_address_v6(const address_v6::bytes_type& bytes,
                                     scope_id_type scope_id);
\end{itemdecl}

\begin{itemdescr}
\pnum
\returns \tcode{address_v6(bytes, scope_id)}.
\end{itemdescr}

\begin{itemdecl}
constexpr address_v6 make_address_v6(v4_mapped_t, const address_v4& a) noexcept;
\end{itemdecl}

\begin{itemdescr}
\pnum
\returns An \tcode{address_v6} object containing the IPv4-mapped IPv6 address corresponding to the specified IPv4 address, as if computed by the following method:
\begin{codeblock}
address_v4::bytes_type v4b = a.to_bytes();
address_v6::bytes_type v6b(0, 0, 0, 0, 0, 0, 0, 0, 0, 0,
                           0xFF, 0xFF, v4b[0], v4b[1], v4b[2], v4b[3]);
return address_v6(v6b);
\end{codeblock}

\end{itemdescr}

\begin{itemdecl}
address_v6 make_address_v6(const char* str);
address_v6 make_address_v6(const char* str, error_code& ec) noexcept;
address_v4 make_address_v6(const string& str);
address_v4 make_address_v6(const string& str, error_code& ec) noexcept;
address_v6 make_address_v6(string_view str);
address_v6 make_address_v6(string_view str, error_code& ec) noexcept;
\end{itemdecl}

\begin{itemdescr}
\pnum
\effects Converts a textual representation of an address into a corresponding \tcode{address_v6} value. The format is either \tcode{address} or \tcode{address}\tcode{\%}\tcode{scope-id}, where \tcode{address} is in the format specified by POSIX \tcode{inet_pton} when invoked with address family \tcode{AF_INET6}, and \tcode{scope-id} is an optional string specifying the scope identifier. All implementations accept as \tcode{scope-id} a textual representation of an unsigned decimal integer. It is \impldef{whether alternative IPv6 scope identifier representations are permitted} whether alternative scope identifier representations are permitted. If \tcode{scope-id} is not supplied, an \tcode{address_v6} object is returned such that \tcode{scope_id() == 0}.

\pnum
\returns If successful, an \tcode{address_v6} value corresponding to the string \tcode{str}. Otherwise returns \tcode{address_v6()}.

\pnum
\errors
\begin{itemize}
\item
\tcode{errc::invalid_argument} --- if \tcode{str} is not a valid textual representation of an IPv6 address.
\end{itemize}
\end{itemdescr}



\rSec2[internet.address.v6.io]{\tcode{ip::address_v6} I/O}

\indexlibrarymember{operator<<}{address_v6}%
\begin{itemdecl}
template<class CharT, class Traits>
  basic_ostream<CharT, Traits>& operator<<(
    basic_ostream<CharT, Traits>& os, const address_v6& addr);
\end{itemdecl}

\begin{itemdescr}
\pnum
\returns \tcode{os << addr.to_string().c_str()}.
\end{itemdescr}




\indexlibrary{\idxcode{bad_address_cast}}%
\rSec1[internet.bad.address.cast]{Class \tcode{ip::bad_address_cast}}

\pnum
An exception of type \tcode{bad_address_cast} is thrown by a failed \tcode{address_cast}.

\begin{codeblock}
namespace std {
namespace experimental {
namespace net {
inline namespace v1 {
namespace ip {

  class bad_address_cast : public bad_cast
  {
  public:
    // constructor:
    bad_address_cast() noexcept;
  };

} // namespace ip
} // inline namespace v1
} // namespace net
} // namespace experimental
} // namespace std
\end{codeblock}

\begin{itemdecl}
bad_address_cast() noexcept;
\end{itemdecl}

\begin{itemdescr}
\pnum
\effects constructs a \tcode{bad_address_cast} object.

\pnum
\postconditions \tcode{what()} returns an \impldef{result of \tcode{bad_address_cast::what}} \ntbs.
\end{itemdescr}



\rSec1[internet.hash]{Hash support}

\begin{itemdecl}
template<> struct hash<experimental::net::v1::ip::address>;
template<> struct hash<experimental::net::v1::ip::address_v4>;
template<> struct hash<experimental::net::v1::ip::address_v6>;
\end{itemdecl}

\begin{itemdescr}
\pnum
\requires the template specializations shall meet the requirements of class template \tcode{hash} (\CppXref{unord.hash}).
\end{itemdescr}



\indexlibrary{\idxcode{basic_address_iterator}}%
\rSec1[internet.address.iter]{Class template \tcode{ip::basic_address_iterator} specializations}

\pnum
The class template \tcode{basic_address_iterator} enables iteration over IP addresses in network byte order. This clause defines two specializations of the class template \tcode{basic_address_iterator}: \tcode{basic_address_iterator<address_v4>} and \tcode{basic_address_iterator<address_v6>}. The members and operational semantics of these specializations are defined below.

\begin{codeblock}
namespace std {
namespace experimental {
namespace net {
inline namespace v1 {
namespace ip {

  template<> class basic_address_iterator<@\placeholder{Address}@>
  {
  public:
    // types:
    using value_type = @\placeholder{Address}@;
    using difference_type = ptrdiff_t;
    using pointer = const @\placeholder{Address}@*;
    using reference = const @\placeholder{Address}@&;
    using iterator_category = input_iterator_tag;

    // constructors:
    basic_address_iterator(const @\placeholder{Address}@& a) noexcept;

    // members:
    reference operator*() const noexcept;
    pointer operator->() const noexcept;
    basic_address_iterator& operator++() noexcept;
    basic_address_iterator operator++(int) noexcept;
    basic_address_iterator& operator--() noexcept;
    basic_address_iterator operator--(int) noexcept;

    // other members as required by \CppXref{input.iterators}

  private:
    @\placeholder{Address}@ address_; // \expos
  };

} // namespace ip
} // inline namespace v1
} // namespace net
} // namespace experimental
} // namespace std
\end{codeblock}

\pnum
Specializations of \tcode{basic_address_iterator} satisfy the requirements for input iterators (\CppXref{input.iterators}).

\begin{itemdecl}
basic_address_iterator(const @\placeholder{Address}@& a) noexcept;
\end{itemdecl}

\begin{itemdescr}
\pnum
\effects Initializes \tcode{address_} with \tcode{a}.
\end{itemdescr}

\begin{itemdecl}
reference operator*() const noexcept;
\end{itemdecl}

\begin{itemdescr}
\pnum
\returns \tcode{address_}.
\end{itemdescr}

\begin{itemdecl}
pointer operator->() const noexcept;
\end{itemdecl}

\begin{itemdescr}
\pnum
\returns \tcode{addressof(address_)}.
\end{itemdescr}

\begin{itemdecl}
basic_address_iterator& operator++() noexcept;
\end{itemdecl}

\begin{itemdescr}
\pnum
\effects Sets \tcode{address_} to the next unique address in network byte order.

\pnum
\returns \tcode{*this}.
\end{itemdescr}

\begin{itemdecl}
basic_address_iterator operator++(int) noexcept;
\end{itemdecl}

\begin{itemdescr}
\pnum
\effects Sets \tcode{address_} to the next unique address in network byte order.

\pnum
\returns The prior value of \tcode{*this}.
\end{itemdescr}

\begin{itemdecl}
basic_address_iterator& operator--() noexcept;
\end{itemdecl}

\begin{itemdescr}
\pnum
\effects Sets \tcode{address_} to the prior unique address in network byte order.

\pnum
\returns \tcode{*this}.
\end{itemdescr}

\begin{itemdecl}
basic_address_iterator operator--(int) noexcept;
\end{itemdecl}

\begin{itemdescr}
\pnum
\effects Sets \tcode{address_} to the prior unique address in network byte order.

\pnum
\returns The prior value of \tcode{*this}.
\end{itemdescr}



\indexlibrary{\idxcode{basic_address_range}}%
\rSec1[internet.address.range]{Class template \tcode{ip::basic_address_range} specializations}

\pnum
The class template \tcode{basic_address_range} represents a range of IP addresses in network byte order. This clause defines two specializations of the class template \tcode{basic_address_range}: \tcode{basic_address_range<address_v4>} and \tcode{basic_address_range<address_v6>}. The members and operational semantics of these specializations are defined below.

\begin{codeblock}
namespace std {
namespace experimental {
namespace net {
inline namespace v1 {
namespace ip {

  template<> class basic_address_range<@\placeholder{Address}@>
  {
  public:
    // types:
    using iterator = basic_address_iterator<@\placeholder{Address}@>;

    // constructors:
    basic_address_range() noexcept;
    basic_address_range(const @\placeholder{Address}@& first,
                        const @\placeholder{Address}@& last) noexcept;

    // members:
    iterator begin() const noexcept;
    iterator end() const noexcept;
    bool empty() const noexcept;
    size_t size() const noexcept; @\textit{// not always defined}@
    iterator find(const @\placeholder{Address}@& addr) const noexcept;
  };

} // namespace ip
} // inline namespace v1
} // namespace net
} // namespace experimental
} // namespace std
\end{codeblock}

\pnum
Specializations of \tcode{basic_address_range} satisfy the requirements for \tcode{Destructible} (\CppXref{destructible}), \tcode{CopyConstructible} (\CppXref{copyconstructible}), and \tcode{CopyAssignable} (\CppXref{copyassignable}).

\begin{itemdecl}
basic_address_range() noexcept;
\end{itemdecl}

\begin{itemdescr}
\pnum
\effects Constructs an object of type \tcode{basic_address_range<\placeholder{Address}>} that represents an empty range.
\end{itemdescr}

\begin{itemdecl}
basic_address_range(const @\placeholder{Address}@& first,
                    const @\placeholder{Address}@& last) noexcept;
\end{itemdecl}

\begin{itemdescr}
\pnum
\effects Constructs an object of type \tcode{basic_address_range<\placeholder{Address}>} that represents the half-open range \range{first}{last}.
\end{itemdescr}

\begin{itemdecl}
iterator begin() const noexcept;
\end{itemdecl}

\begin{itemdescr}
\pnum
\returns An iterator that points to the beginning of the range.
\end{itemdescr}

\begin{itemdecl}
iterator end() const noexcept;
\end{itemdecl}

\begin{itemdescr}
\pnum
\returns An iterator that points to the end of the range.
\end{itemdescr}

\begin{itemdecl}
bool empty() const noexcept;
\end{itemdecl}

\begin{itemdescr}
\pnum
\returns \tcode{true} if \tcode{*this} represents an empty range, otherwise \tcode{false}.
\end{itemdescr}

\begin{itemdecl}
size_t size() const noexcept;
\end{itemdecl}

\begin{itemdescr}
\pnum
\returns The number of unique addresses in the range.

\pnum
\remarks This member function is not defined when \placeholder{Address} is type \tcode{address_v6}.
\end{itemdescr}

\begin{itemdecl}
iterator find(const @\placeholder{Address}@& addr) const noexcept;
\end{itemdecl}

\begin{itemdescr}
\pnum
\returns If \tcode{addr} is in the range, an iterator that points to \tcode{addr}; otherwise, \tcode{end()}.

\pnum
Complexity: Constant time.
\end{itemdescr}



\indexlibrary{\idxcode{network_v4}}%
\rSec1[internet.network.v4]{Class template \tcode{ip::network_v4}}

\pnum
The class \tcode{network_v4} provides the ability to use and manipulate IPv4 network addresses.

\begin{codeblock}
namespace std {
namespace experimental {
namespace net {
inline namespace v1 {
namespace ip {

  class network_v4
  {
  public:
    // \ref{internet.network.v4.cons}, constructors:
    constexpr network_v4() noexcept;
    constexpr network_v4(const address_v4& addr, int prefix_len);
    constexpr network_v4(const address_v4& addr, const address_v4& mask);

    // \ref{internet.network.v4.members}, members:
    constexpr address_v4 address() const noexcept;
    constexpr int prefix_length() const noexcept;
    constexpr address_v4 netmask() const noexcept;
    constexpr address_v4 network() const noexcept;
    constexpr address_v4 broadcast() const noexcept;
    address_v4_range hosts() const noexcept;
    constexpr network_v4 canonical() const noexcept;
    constexpr bool is_host() const noexcept;
    constexpr bool is_subnet_of(const network_v4& other) const noexcept;
    template<class Allocator = allocator<char>>
      basic_string<char, char_traits<char>, Allocator>
        to_string(const Allocator& a = Allocator()) const;
  };

  // \ref{internet.network.v4.comparisons}, network_v4 comparisons:
  constexpr bool operator==(const network_v4& a, const network_v4& b) noexcept;
  constexpr bool operator!=(const network_v4& a, const network_v4& b) noexcept;

  // \ref{internet.network.v4.creation}, network_v4 creation:
  constexpr network_v4 make_network_v4(const address_v4& addr, int prefix_len);
  constexpr network_v4 make_network_v4(const address_v4& addr, const address_v4& mask);
  network_v4 make_network_v4(const char* str);
  network_v4 make_network_v4(const char* str, error_code& ec) noexcept;
  network_v4 make_network_v4(const string& str);
  network_v4 make_network_v4(const string& str, error_code& ec) noexcept;
  network_v4 make_network_v4(string_view str);
  network_v4 make_network_v4(string_view str, error_code& ec) noexcept;

  // \ref{internet.network.v4.io}, network_v4 I/O:
  template<class CharT, class Traits>
    basic_ostream<CharT, Traits>& operator<<(
      basic_ostream<CharT, Traits>& os, const network_v4& net);

} // namespace ip
} // inline namespace v1
} // namespace net
} // namespace experimental
} // namespace std
\end{codeblock}

\pnum
\tcode{network_v4} satisfies the requirements for \tcode{Destructible} (\CppXref{destructible}), \tcode{CopyConstructible} (\CppXref{copyconstructible}), and \tcode{CopyAssignable} (\CppXref{copyassignable}).


\rSec2[internet.network.v4.cons]{\tcode{ip::network_v4} constructors}

\begin{itemdecl}
constexpr network_v4() noexcept;
\end{itemdecl}

\begin{itemdescr}
\pnum
\postconditions \tcode{this->address().is_unspecified() == true} and \tcode{prefix_length() == 0}.
\end{itemdescr}

\begin{itemdecl}
constexpr network_v4(const address_v4& addr, int prefix_len);
\end{itemdecl}

\begin{itemdescr}
\pnum
\postconditions \tcode{this->address() == addr} and \tcode{prefix_length() == prefix_len}.

\pnum
\remarks \tcode{out_of_range} if \tcode{prefix_len < 0} or \tcode{prefix_len > 32}.
\end{itemdescr}

\begin{itemdecl}
constexpr network_v4(const address_v4& addr, const address_v4& mask);
\end{itemdecl}

\begin{itemdescr}
\pnum
\postconditions \tcode{this->address() == addr} and \tcode{prefix_length()} is equal to the number of contiguous non-zero bits in \tcode{mask}.

\pnum
\remarks \tcode{invalid_argument} if \tcode{mask} contains non-contiguous non-zero bits, or if the most significant bit is zero and any other bits are non-zero.
\end{itemdescr}



\rSec2[internet.network.v4.members]{\tcode{ip::network_v4} members}

\begin{itemdecl}
constexpr address_v4 address() const noexcept;
\end{itemdecl}

\begin{itemdescr}
\pnum
\returns The address specified when the \tcode{network_v4} object was constructed.
\end{itemdescr}

\begin{itemdecl}
constexpr int prefix_length() const noexcept;
\end{itemdecl}

\begin{itemdescr}
\pnum
\returns The prefix length of the network.
\end{itemdescr}

\begin{itemdecl}
constexpr address_v4 netmask() const noexcept;
\end{itemdecl}

\begin{itemdescr}
\pnum
\returns An \tcode{address_v4} object with \tcode{prefix_length()} contiguous non-zero bits set, starting from the most significant bit in network byte order. All other bits are zero.
\end{itemdescr}

\begin{itemdecl}
constexpr address_v4 network() const noexcept;
\end{itemdecl}

\begin{itemdescr}
\pnum
\returns An \tcode{address_v4} object with the first \tcode{prefix_length()} bits, starting from the most significant bit in network byte order, set to the corresponding bit value of \tcode{this->address()}. All other bits are zero.
\end{itemdescr}

\begin{itemdecl}
constexpr address_v4 broadcast() const noexcept;
\end{itemdecl}

\begin{itemdescr}
\pnum
\returns An \tcode{address_v4} object with the first \tcode{prefix_length()} bits, starting from the most significant bit in network byte order, set to the corresponding bit value of \tcode{this->address()}. All other bits are non-zero.
\end{itemdescr}

\begin{itemdecl}
address_v4_range hosts() const noexcept;
\end{itemdecl}

\begin{itemdescr}
\pnum
\returns If \tcode{is_host()} is \tcode{true}, an \tcode{address_v4_range} object representing the single address \tcode{this->add\-r\-ess()}. Otherwise, an \tcode{address_v4_range} object representing the range of unique host IP addresses in the network.

\pnum
\begin{note} For IPv4, the network address and the broadcast address are not included in the range of host IP addresses. For example, given a network \tcode{192.168.1.0/24}, the range returned by \tcode{hosts()} is from \tcode{192.168.1.1} to \tcode{192.168.1.254} inclusive, and neither \tcode{192.168.1.0} nor the broadcast address \tcode{192.168.1.255} are in the range. \end{note}
\end{itemdescr}

\begin{itemdecl}
constexpr network_v4 canonical() const noexcept;
\end{itemdecl}

\begin{itemdescr}
\pnum
\returns \tcode{network_v4(network(), prefix_length())}.
\end{itemdescr}

\begin{itemdecl}
constexpr bool is_host() const noexcept;
\end{itemdecl}

\begin{itemdescr}
\pnum
\returns \tcode{prefix_length() == 32}.
\end{itemdescr}

\begin{itemdecl}
constexpr bool is_subnet_of(const network_v4& other) const noexcept;
\end{itemdecl}

\begin{itemdescr}
\pnum
\returns \tcode{true} if \tcode{other.prefix_length() < prefix_length()} and \tcode{network_v4(this->address(), other.prefix_length()).canonical() == other.canonical()}, otherwise \tcode{false}.
\end{itemdescr}

\begin{itemdecl}
template<class Allocator = allocator<char>>
  basic_string<char, char_traits<char>, Allocator>
    to_string(const Allocator& a = Allocator()) const;
\end{itemdecl}

\begin{itemdescr}
\pnum
\returns \tcode{this->address().to_string(a) + "/" + std::to_string(prefix_length())}.
\end{itemdescr}



\rSec2[internet.network.v4.comparisons]{\tcode{ip::network_v4} comparisons}

\begin{itemdecl}
constexpr bool operator==(const network_v4& a, const network_v4& b) noexcept;
\end{itemdecl}

\begin{itemdescr}
\pnum
\returns \tcode{true} if \tcode{a.address() == b.address()} and \tcode{a.prefix_length() == b.prefix_length()}, otherwise \tcode{false}.
\end{itemdescr}

\begin{itemdecl}
constexpr bool operator!=(const network_v4& a, const network_v4& b) noexcept;
\end{itemdecl}

\begin{itemdescr}
\pnum
\returns \tcode{!(a == b)}.
\end{itemdescr}



\rSec2[internet.network.v4.creation]{\tcode{ip::network_v4} creation}

\begin{itemdecl}
constexpr network_v4 make_network_v4(const address_v4& addr, int prefix_len);
\end{itemdecl}

\begin{itemdescr}
\pnum
\returns \tcode{network_v4(addr, prefix_len)}.
\end{itemdescr}

\begin{itemdecl}
constexpr network_v4 make_network_v4(const address_v4& addr, const address_v4& mask);
\end{itemdecl}

\begin{itemdescr}
\pnum
\returns \tcode{network_v4(addr, mask)}.
\end{itemdescr}

\begin{itemdecl}
network_v4 make_network_v4(const char* str);
network_v4 make_network_v4(const char* str, error_code& ec) noexcept;
network_v4 make_network_v4(const string& str);
network_v4 make_network_v4(const string& str, error_code& ec) noexcept;
network_v4 make_network_v4(string_view str);
network_v4 make_network_v4(string_view str, error_code& ec) noexcept;
\end{itemdecl}

\begin{itemdescr}
\pnum
\returns If \tcode{str} contains a value of the form address \tcode{'/'} prefix-length, a \tcode{network_v4} object constructed with the result of applying \tcode{make_address_v4()} to the address portion of the string, and the result of converting prefix-length to an integer of type \tcode{int}. Otherwise returns \tcode{network_v4()} and sets \tcode{ec} to reflect the error.

\pnum
\errors
\begin{itemize}
\item
\tcode{errc::invalid_argument} --- if \tcode{str} is not a valid textual representation of an IPv4 address and prefix length.
\end{itemize}
\end{itemdescr}



\rSec2[internet.network.v4.io]{\tcode{ip::network_v4} I/O}

\begin{itemdecl}
template<class CharT, class Traits>
  basic_ostream<CharT, Traits>& operator<<(
    basic_ostream<CharT, Traits>& os, const network_v4& net);
\end{itemdecl}

\begin{itemdescr}
\pnum
\returns \tcode{os << net.to_string().c_str()}.
\end{itemdescr}




\indexlibrary{\idxcode{network_v6}}%
\rSec1[internet.network.v6]{Class template \tcode{ip::network_v6}}

\pnum
The class \tcode{network_v6} provides the ability to use and manipulate IPv6 network addresses.

\begin{codeblock}
namespace std {
namespace experimental {
namespace net {
inline namespace v1 {
namespace ip {

  class network_v6
  {
  public:
    // \ref{internet.network.v6.cons}, constructors:
    constexpr network_v6() noexcept;
    constexpr network_v6(const address_v6& addr, int prefix_len);

    // \ref{internet.network.v6.members}, members:
    constexpr address_v6 address() const noexcept;
    constexpr int prefix_length() const noexcept;
    constexpr address_v6 network() const noexcept;
    address_v6_range hosts() const noexcept;
    constexpr network_v6 canonical() const noexcept;
    constexpr bool is_host() const noexcept;
    constexpr bool is_subnet_of(const network_v6& other) const noexcept;
    template<class Allocator = allocator<char>>
      basic_string<char, char_traits<char>, Allocator>
        to_string(const Allocator& a = Allocator()) const;
  };

  // \ref{internet.network.v6.comparisons}, network_v6 comparisons:
  constexpr bool operator==(const network_v6& a, const network_v6& b) noexcept;
  constexpr bool operator!=(const network_v6& a, const network_v6& b) noexcept;

  // \ref{internet.network.v6.creation}, network_v6 creation:
  constexpr network_v6 make_network_v6(const address_v6& addr, int prefix_len);
  network_v6 make_network_v6(const char* str);
  network_v6 make_network_v6(const char* str, error_code& ec) noexcept;
  network_v6 make_network_v6(const string& str);
  network_v6 make_network_v6(const string& str, error_code& ec) noexcept;
  network_v6 make_network_v6(string_view str);
  network_v6 make_network_v6(string_view str, error_code& ec) noexcept;

  // \ref{internet.network.v6.io}, network_v6 I/O:
  template<class CharT, class Traits>
    basic_ostream<CharT, Traits>& operator<<(
      basic_ostream<CharT, Traits>& os, const network_v6& net);

} // namespace ip
} // inline namespace v1
} // namespace net
} // namespace experimental
} // namespace std
\end{codeblock}

\pnum
\tcode{network_v6} satisfies the requirements for \tcode{Destructible} (\CppXref{destructible}), \tcode{CopyConstructible} (\CppXref{copyconstructible}), and \tcode{CopyAssignable} (\CppXref{copyassignable}).


\rSec2[internet.network.v6.cons]{\tcode{ip::network_v6} constructors}

\begin{itemdecl}
constexpr network_v6() noexcept;
\end{itemdecl}

\begin{itemdescr}
\pnum
\postconditions \tcode{this->address().is_unspecified() == true} and \tcode{prefix_length() == 0}.
\end{itemdescr}

\begin{itemdecl}
constexpr network_v6(const address_v6& addr, int prefix_len);
\end{itemdecl}

\begin{itemdescr}
\pnum
\postconditions \tcode{this->address() == addr} and \tcode{prefix_length() == prefix_len}.

\pnum
\remarks \tcode{out_of_range} if \tcode{prefix_len < 0} or \tcode{prefix_len > 128}.
\end{itemdescr}



\rSec2[internet.network.v6.members]{\tcode{ip::network_v6} members}

\begin{itemdecl}
constexpr address_v6 address() const noexcept;
\end{itemdecl}

\begin{itemdescr}
\pnum
\returns The address specified when the \tcode{network_v6} object was constructed.
\end{itemdescr}

\begin{itemdecl}
constexpr int prefix_length() const noexcept;
\end{itemdecl}

\begin{itemdescr}
\pnum
\returns The prefix length of the network.
\end{itemdescr}

\begin{itemdecl}
constexpr address_v6 network() const noexcept;
\end{itemdecl}

\begin{itemdescr}
\pnum
\returns An \tcode{address_v6} object with the first \tcode{prefix_length()} bits, starting from the most significant bit in network byte order, set to the corresponding bit value of \tcode{this->address()}. All other bits are zero.
\end{itemdescr}

\begin{itemdecl}
address_v6_range hosts() const noexcept;
\end{itemdecl}

\begin{itemdescr}
\pnum
\returns If \tcode{is_host()} is \tcode{true}, an \tcode{address_v6_range} object representing the single address \tcode{this->addr\-ess()}. Otherwise, an \tcode{address_v6_range} object representing the range of unique host IP addresses in the network.
\end{itemdescr}

\begin{itemdecl}
constexpr network_v6 canonical() const noexcept;
\end{itemdecl}

\begin{itemdescr}
\pnum
\returns \tcode{network_v6(network(), prefix_length())}.
\end{itemdescr}

\begin{itemdecl}
constexpr bool is_host() const noexcept;
\end{itemdecl}

\begin{itemdescr}
\pnum
\returns \tcode{prefix_length() == 128}.
\end{itemdescr}

\begin{itemdecl}
constexpr bool is_subnet_of(const network_v6& other) const noexcept;
\end{itemdecl}

\begin{itemdescr}
\pnum
\returns \tcode{true} if \tcode{other.prefix_length() < prefix_length()} and \tcode{network_v6(this->address(), other.prefix_length()).canonical() == other.canonical()}, otherwise \tcode{false}.
\end{itemdescr}

\begin{itemdecl}
template<class Allocator = allocator<char>>
  basic_string<char, char_traits<char>, Allocator>
    to_string(const Allocator& a = Allocator()) const;
\end{itemdecl}

\begin{itemdescr}
\pnum
\returns \tcode{this->address().to_string(a) + "/" + to_string(prefix_length()).c_str()}.
\end{itemdescr}



\rSec2[internet.network.v6.comparisons]{\tcode{ip::network_v6} comparisons}

\begin{itemdecl}
constexpr bool operator==(const network_v6& a, const network_v6& b) noexcept;
\end{itemdecl}

\begin{itemdescr}
\pnum
\returns \tcode{true} if \tcode{a.address() == b.address()} and \tcode{a.prefix_length() == b.prefix_length()}, otherwise \tcode{false}.
\end{itemdescr}

\begin{itemdecl}
constexpr bool operator!=(const network_v6& a, const network_v6& b) noexcept;
\end{itemdecl}

\begin{itemdescr}
\pnum
\returns \tcode{!(a == b)}.
\end{itemdescr}



\rSec2[internet.network.v6.creation]{\tcode{ip::network_v6} creation}

\begin{itemdecl}
constexpr network_v6 make_network_v6(const address_v6& addr, int prefix_len);
\end{itemdecl}

\begin{itemdescr}
\pnum
\returns \tcode{network_v6(addr, prefix_len)}.
\end{itemdescr}

\begin{itemdecl}
network_v6 make_network_v6(const char* str);
network_v6 make_network_v6(const char* str, error_code& ec) noexcept;
network_v6 make_network_v6(const string& str);
network_v6 make_network_v6(const string& str, error_code& ec) noexcept;
network_v6 make_network_v6(string_view str);
network_v6 make_network_v6(string_view str, error_code& ec) noexcept;
\end{itemdecl}

\begin{itemdescr}
\pnum
\returns If \tcode{str} contains a value of the form address \tcode{'/'} prefix-length, a \tcode{network_v6} object constructed with the result of applying \tcode{make_address_v6()} to the address portion of the string, and the result of converting prefix-length to an integer of type \tcode{int}. Otherwise returns \tcode{network_v6()} and sets \tcode{ec} to reflect the error.

\pnum
\errors
\begin{itemize}
\item
\tcode{errc::invalid_argument} --- if \tcode{str} is not a valid textual representation of an IPv6 address and prefix length.
\end{itemize}
\end{itemdescr}



\rSec2[internet.network.v6.io]{\tcode{ip::network_v6} I/O}

\begin{itemdecl}
template<class CharT, class Traits>
  basic_ostream<CharT, Traits>& operator<<(
    basic_ostream<CharT, Traits>& os, const network_v6& net);
\end{itemdecl}

\begin{itemdescr}
\pnum
\returns \tcode{os << net.to_string().c_str()}.
\end{itemdescr}




\indexlibrary{\idxcode{basic_endpoint}}%
\rSec1[internet.endpoint]{Class template \tcode{ip::basic_endpoint}}

\pnum
An object of type \tcode{basic_endpoint<InternetProtocol>} represents a protocol-specific endpoint, where an endpoint consists of an IP address and port number. Endpoints may be used to identify sources and destinations for socket connections and datagrams.

\begin{codeblock}
namespace std {
namespace experimental {
namespace net {
inline namespace v1 {
namespace ip {

  template<class InternetProtocol>
  class basic_endpoint
  {
  public:
    // types:
    using protocol_type = InternetProtocol;

    // \ref{internet.endpoint.cons}, constructors:
    constexpr basic_endpoint() noexcept;
    constexpr basic_endpoint(const protocol_type& proto,
                             port_type port_num) noexcept;
    constexpr basic_endpoint(const ip::address& addr,
                             port_type port_num) noexcept;

    // \ref{internet.endpoint.members}, members:
    constexpr protocol_type protocol() const noexcept;
    constexpr ip::address address() const noexcept;
    void address(const ip::address& addr) noexcept;
    constexpr port_type port() const noexcept;
    void port(port_type port_num) noexcept;
  };

  // \ref{internet.endpoint.comparisons}, basic_endpoint comparisons:
  template<class InternetProtocol>
    constexpr bool operator==(const basic_endpoint<InternetProtocol>& a,
                              const basic_endpoint<InternetProtocol>& b) noexcept;
  template<class InternetProtocol>
    constexpr bool operator!=(const basic_endpoint<InternetProtocol>& a,
                              const basic_endpoint<InternetProtocol>& b) noexcept;
  template<class InternetProtocol>
    constexpr bool operator< (const basic_endpoint<InternetProtocol>& a,
                              const basic_endpoint<InternetProtocol>& b) noexcept;
  template<class InternetProtocol>
    constexpr bool operator> (const basic_endpoint<InternetProtocol>& a,
                              const basic_endpoint<InternetProtocol>& b) noexcept;
  template<class InternetProtocol>
    constexpr bool operator<=(const basic_endpoint<InternetProtocol>& a,
                              const basic_endpoint<InternetProtocol>& b) noexcept;
  template<class InternetProtocol>
    constexpr bool operator>=(const basic_endpoint<InternetProtocol>& a,
                              const basic_endpoint<InternetProtocol>& b) noexcept;

  // \ref{internet.endpoint.io}, basic_endpoint I/O:
  template<class CharT, class Traits, class InternetProtocol>
    basic_ostream<CharT, Traits>& operator<<(
      basic_ostream<CharT, Traits>& os,
      const basic_endpoint<InternetProtocol>& ep);

} // namespace ip
} // inline namespace v1
} // namespace net
} // namespace experimental
} // namespace std
\end{codeblock}

\pnum
Instances of the \tcode{basic_endpoint} class template meet the requirements for an \tcode{Endpoint}~(\ref{socket.reqmts.endpoint}).

\indexextensible{basic_endpoint}%
\pnum
 Extensible implementations provide the following member functions:

\begin{codeblock}
namespace std {
namespace experimental {
namespace net {
inline namespace v1 {
namespace ip {

  template<class InternetProtocol>
  class basic_endpoint
  {
  public:
    void* data() noexcept;
    const void* data() const noexcept;
    constexpr size_t size() const noexcept;
    void resize(size_t s);
    constexpr size_t capacity() const noexcept;
    @\textit{// remainder unchanged}@
  private:
    union
    {
      sockaddr_in v4_;
      sockaddr_in6 v6_;
    } data_; // \expos
  };

} // namespace ip
} // inline namespace v1
} // namespace net
} // namespace experimental
} // namespace std
\end{codeblock}


\rSec2[internet.endpoint.cons]{\tcode{ip::basic_endpoint} constructors}

\begin{itemdecl}
constexpr basic_endpoint() noexcept;
\end{itemdecl}

\begin{itemdescr}
\pnum
\postconditions \tcode{this->address() == ip::address()} and \tcode{port() == 0}.
\end{itemdescr}

\begin{itemdecl}
constexpr basic_endpoint(const protocol_type& proto,
                         port_type port_num) noexcept;
\end{itemdecl}

\begin{itemdescr}
\pnum
\requires \tcode{proto == protocol_type::v4() || proto == protocol_type::v6()}.

\pnum
\postconditions
\begin{itemize}
\item
 If \tcode{proto == protocol_type::v6()}, \tcode{this->address() == ip::address_v6()}; otherwise, \tcode{this->address() == ip::address_v4()}.
\item
\tcode{port() == port_num}.
\end{itemize}
\end{itemdescr}

\begin{itemdecl}
constexpr basic_endpoint(const ip::address& addr,
                         port_type port_num) noexcept;
\end{itemdecl}

\begin{itemdescr}
\pnum
\postconditions \tcode{this->address() == addr} and \tcode{port() == port_num}.
\end{itemdescr}



\rSec2[internet.endpoint.members]{\tcode{ip::basic_endpoint} members}

\begin{itemdecl}
constexpr protocol_type protocol() const noexcept;
\end{itemdecl}

\begin{itemdescr}
\pnum
\returns \tcode{protocol_type::v6()} if the expression \tcode{this->address().is_v6()} is \tcode{true}, otherwise \tcode{protocol_type::v4()}.
\end{itemdescr}

\begin{itemdecl}
constexpr ip::address address() const noexcept;
\end{itemdecl}

\begin{itemdescr}
\pnum
\returns The address associated with the endpoint.
\end{itemdescr}

\begin{itemdecl}
void address(const ip::address& addr) noexcept;
\end{itemdecl}

\begin{itemdescr}
\pnum
\postconditions \tcode{this->address() == addr}.
\end{itemdescr}

\begin{itemdecl}
constexpr port_type port() const noexcept;
\end{itemdecl}

\begin{itemdescr}
\pnum
\returns The port number associated with the endpoint.
\end{itemdescr}

\begin{itemdecl}
void port(port_type port_num) noexcept;
\end{itemdecl}

\begin{itemdescr}
\pnum
\postconditions \tcode{port() == port_num}.
\end{itemdescr}



\rSec2[internet.endpoint.comparisons]{\tcode{ip::basic_endpoint} comparisons}

\begin{itemdecl}
template<class InternetProtocol>
  constexpr bool operator==(const basic_endpoint<InternetProtocol>& a,
                            const basic_endpoint<InternetProtocol>& b) noexcept;
\end{itemdecl}

\begin{itemdescr}
\pnum
\returns \tcode{a.address() == b.address() \&\& a.port() == b.port())}.
\end{itemdescr}

\begin{itemdecl}
template<class InternetProtocol>
  constexpr bool operator!=(const basic_endpoint<InternetProtocol>& a,
                            const basic_endpoint<InternetProtocol>& b) noexcept;
\end{itemdecl}

\begin{itemdescr}
\pnum
\returns \tcode{!(a == b)}.
\end{itemdescr}

\begin{itemdecl}
template<class InternetProtocol>
  constexpr bool operator< (const basic_endpoint<InternetProtocol>& a,
                            const basic_endpoint<InternetProtocol>& b) noexcept;
\end{itemdecl}

\begin{itemdescr}
\pnum
\returns \tcode{tie(a.address(), a.port()) < tie(b.address(), b.port())}.
\end{itemdescr}

\begin{itemdecl}
template<class InternetProtocol>
  constexpr bool operator> (const basic_endpoint<InternetProtocol>& a,
                            const basic_endpoint<InternetProtocol>& b) noexcept;
\end{itemdecl}

\begin{itemdescr}
\pnum
\returns \tcode{b < a}.
\end{itemdescr}

\begin{itemdecl}
template<class InternetProtocol>
  constexpr bool operator<=(const basic_endpoint<InternetProtocol>& a,
                            const basic_endpoint<InternetProtocol>& b) noexcept;
\end{itemdecl}

\begin{itemdescr}
\pnum
\returns \tcode{!(b < a)}.
\end{itemdescr}

\begin{itemdecl}
template<class InternetProtocol>
  constexpr bool operator>=(const basic_endpoint<InternetProtocol>& a,
                            const basic_endpoint<InternetProtocol>& b) noexcept;
\end{itemdecl}

\begin{itemdescr}
\pnum
\returns \tcode{!(a < b)}.
\end{itemdescr}



\rSec2[internet.endpoint.io]{\tcode{ip::basic_endpoint} I/O}

\begin{itemdecl}
template<class CharT, class Traits, class InternetProtocol>
  basic_ostream<CharT, Traits>& operator<<(
    basic_ostream<CharT, Traits>& os,
    const basic_endpoint<InternetProtocol>& ep);
\end{itemdecl}

\begin{itemdescr}
\pnum
\effects Outputs a representation of the endpoint to the stream, as if it were implemented as follows:
\begin{codeblock}
basic_ostringstream<CharT, Traits> ss;
if (ep.protocol() == basic_endpoint<InternetProtocol>::protocol_type::v6())
  ss << "[" << ep.address() << "]";
else
  ss << ep.address();
ss << ":" << ep.port();
os << ss.str();
\end{codeblock}


\pnum
\returns \tcode{os}.
\end{itemdescr}

\pnum
\begin{note} The representation of the endpoint when it contains an IP version 6 address is based on [RFC2732]. \end{note}


\rSec2[internet.endpoint.extensible]{\tcode{ip::basic_endpoint} members (extensible implementations)}

\begin{itemdecl}
void* data() noexcept;
\end{itemdecl}

\begin{itemdescr}
\pnum
\returns \tcode{addressof(data_)}.
\end{itemdescr}

\begin{itemdecl}
const void* data() const noexcept;
\end{itemdecl}

\begin{itemdescr}
\pnum
\returns \tcode{addressof(data_)}.
\end{itemdescr}

\begin{itemdecl}
constexpr size_t size() const noexcept;
\end{itemdecl}

\begin{itemdescr}
\pnum
\returns \tcode{sizeof(sockaddr_in6)} if \tcode{protocol().family() == AF_INET6}, otherwise \tcode{sizeof(sock\-addr_in)}.
\end{itemdescr}

\begin{itemdecl}
void resize(size_t s);
\end{itemdecl}

\begin{itemdescr}
\pnum
\remarks \tcode{length_error} if either of the following conditions is \tcode{true}:
\begin{itemize}
\item \tcode{protocol().family() == AF_INET6 \&\& s != sizeof(sock\-addr_in6)},
\item \tcode{protocol().family() == AF_INET4 \&\& s != sizeof(sockaddr_in)}.
\end{itemize}
\end{itemdescr}

\begin{itemdecl}
constexpr size_t capacity() const noexcept;
\end{itemdecl}

\begin{itemdescr}
\pnum
\returns \tcode{sizeof(data_)}.
\end{itemdescr}




\indexlibrary{\idxcode{basic_resolver_entry}}%
\rSec1[internet.resolver.entry]{Class template \tcode{ip::basic_resolver_entry}}

\pnum
An object of type \tcode{basic_resolver_entry<InternetProtocol>} represents a single element in the results returned by a name resolution operation.

\begin{codeblock}
namespace std {
namespace experimental {
namespace net {
inline namespace v1 {
namespace ip {

  template<class InternetProtocol>
  class basic_resolver_entry
  {
  public:
    // types:
    using protocol_type = InternetProtocol;
    using endpoint_type = typename InternetProtocol::endpoint;

    // \ref{internet.resolver.entry.cons}, constructors:
    basic_resolver_entry();
    basic_resolver_entry(const endpoint_type& ep,
                         string_view h,
                         string_view s);

    // \ref{internet.resolver.entry.members}, members:
    endpoint_type endpoint() const;
    operator endpoint_type() const;
    template<class Allocator = allocator<char>>
      basic_string<char, char_traits<char>, Allocator>
        host_name(const Allocator& a = Allocator()) const;
    template<class Allocator = allocator<char>>
      basic_string<char, char_traits<char>, Allocator>
        service_name(const Allocator& a = Allocator()) const;
  };

  // \ref{internet.resolver.entry.comparisons}, basic_resolver_entry comparisons:
  template<class InternetProtocol>
    bool operator==(const basic_resolver_entry<InternetProtocol>& a,
                    const basic_resolver_entry<InternetProtocol>& b);
  template<class InternetProtocol>
    bool operator!=(const basic_resolver_entry<InternetProtocol>& a,
                    const basic_resolver_entry<InternetProtocol>& b);

} // namespace ip
} // inline namespace v1
} // namespace net
} // namespace experimental
} // namespace std
\end{codeblock}


\rSec2[internet.resolver.entry.cons]{\tcode{ip::basic_resolver_entry} constructors}

\begin{itemdecl}
basic_resolver_entry();
\end{itemdecl}

\begin{itemdescr}
\pnum
\effects Equivalent to \tcode{basic_resolver_entry<InternetProtocol>(endpoint_type(), "", "")}.
\end{itemdescr}

\begin{itemdecl}
basic_resolver_entry(const endpoint_type& ep,
                     string_view h,
                     string_view s);
\end{itemdecl}

\begin{itemdescr}
\pnum
\postconditions 
\begin{itemize}
\item
\tcode{endpoint() == ep}.
\item
\tcode{host_name() == h}.
\item
\tcode{service_name() == s}.
\end{itemize}
\end{itemdescr}



\rSec2[internet.resolver.entry.members]{\tcode{ip::basic_resolver_entry} members}

\begin{itemdecl}
endpoint_type endpoint() const;
\end{itemdecl}

\begin{itemdescr}
\pnum
\returns The endpoint associated with the resolver entry.
\end{itemdescr}

\begin{itemdecl}
operator endpoint_type() const;
\end{itemdecl}

\begin{itemdescr}
\pnum
\returns \tcode{endpoint()}.
\end{itemdescr}

\begin{itemdecl}
template<class Allocator = allocator<char>>
  basic_string<char, char_traits<char>, Allocator>
    host_name(const Allocator& a = Allocator()) const;
\end{itemdecl}

\begin{itemdescr}
\pnum
\returns The host name associated with the resolver entry.

\pnum
\remarks Ill-formed unless \tcode{allocator_traits<Allocator>::value_type} is \tcode{char}.
\end{itemdescr}

\begin{itemdecl}
template<class Allocator = allocator<char>>
  basic_string<char, char_traits<char>, Allocator>
    service_name(const Allocator& a = Allocator()) const;
\end{itemdecl}

\begin{itemdescr}
\pnum
\returns The service name associated with the resolver entry.

\pnum
\remarks Ill-formed unless \tcode{allocator_traits<Allocator>::value_type} is \tcode{char}.
\end{itemdescr}



\rSec2[internet.resolver.entry.comparisons]{\tcode{op::basic_resolver_entry} comparisons}

\begin{itemdecl}
template<class InternetProtocol>
  bool operator==(const basic_resolver_entry<InternetProtocol>& a,
                  const basic_resolver_entry<InternetProtocol>& b);
\end{itemdecl}

\begin{itemdescr}
\pnum
\returns \tcode{a.endpoint() == b.endpoint() \&\& a.host_name() == b.host_name() \&\& a.service_name() == b.service_name()}.
\end{itemdescr}

\begin{itemdecl}
template<class InternetProtocol>
  bool operator!=(const basic_resolver_entry<InternetProtocol>& a,
                  const basic_resolver_entry<InternetProtocol>& b);
\end{itemdecl}

\begin{itemdescr}
\pnum
\returns \tcode{!(a == b)}.
\end{itemdescr}




\indexlibrary{\idxcode{basic_resolver_results}}%
\rSec1[internet.resolver.results]{Class template \tcode{ip::basic_resolver_results}}

\pnum
An object of type \tcode{basic_resolver_results<InternetProtocol>}
represents a sequence of \tcode{basic_res\-olver_entry<InternetProtocol>} elements
resulting from a single name resolution operation.

\begin{codeblock}
namespace std {
namespace experimental {
namespace net {
inline namespace v1 {
namespace ip {

  template<class InternetProtocol>
  class basic_resolver_results
  {
  public:
    // types:
    using protocol_type = InternetProtocol;
    using endpoint_type = typename protocol_type::endpoint;
    using value_type = basic_resolver_entry<protocol_type>;
    using const_reference = const value_type&;
    using reference = value_type&;
    using const_iterator = @\impdefx{type of \tcode{basic_resolver_results::const_iterator}}@;
    using iterator = const_iterator;
    using difference_type = ptrdiff_t;
    using size_type = size_t;

    // \ref{internet.resolver.results.cons}, construct / copy / destroy:
    basic_resolver_results();
    basic_resolver_results(const basic_resolver_results& rhs);
    basic_resolver_results(basic_resolver_results&& rhs) noexcept;
    basic_resolver_results& operator=(const basic_resolver_results& rhs);
    basic_resolver_results& operator=(basic_resolver_results&& rhs);
    ~basic_resolver_results();

    // \ref{internet.resolver.results.size}, size:
    size_type size() const noexcept;
    size_type max_size() const noexcept;
    bool empty() const noexcept;

    // \ref{internet.resolver.results.access}, element access:
    const_iterator begin() const;
    const_iterator end() const;
    const_iterator cbegin() const;
    const_iterator cend() const;

    // \ref{internet.resolver.results.swap}, swap:
    void swap(basic_resolver_results& that) noexcept;
  };

  // \ref{internet.resolver.results.comparisons}, basic_resolver_results comparisons:
  template<class InternetProtocol>
    bool operator==(const basic_resolver_results<InternetProtocol>& a,
                    const basic_resolver_results<InternetProtocol>& b);
  template<class InternetProtocol>
    bool operator!=(const basic_resolver_results<InternetProtocol>& a,
                    const basic_resolver_results<InternetProtocol>& b);

} // namespace ip
} // inline namespace v1
} // namespace net
} // namespace experimental
} // namespace std
\end{codeblock}

\pnum
The class template \tcode{basic_resolver_results} satisfies the requirements of a sequence container (\CppXref{sequence.reqmts}), except that only the operations defined for const-qualified sequence containers are supported. The class template \tcode{basic_resolver_results} supports forward iterators.

\pnum
A default-constructed \tcode{basic_resolver_results} object is empty. A non-empty results object is obtained only by calling a \tcode{basic_resolver} object's \tcode{wait} or \tcode{async_wait} operations, or otherwise by copy construction, move construction, assignment, or swap from another non-empty results object.


\rSec2[internet.resolver.results.cons]{\tcode{ip::basic_resolver_results} constructors}

\begin{itemdecl}
basic_resolver_results();
\end{itemdecl}

\begin{itemdescr}
\pnum
\postconditions \tcode{size() == 0}.
\end{itemdescr}

\begin{itemdecl}
basic_resolver_results(const basic_resolver_results& rhs);
\end{itemdecl}

\begin{itemdescr}
\pnum
\postconditions \tcode{*this == rhs}.
\end{itemdescr}

\begin{itemdecl}
basic_resolver_results(basic_resolver_results&& rhs) noexcept;
\end{itemdecl}

\begin{itemdescr}
\pnum
\postconditions \tcode{*this} is equal to the prior value of \tcode{rhs}.
\end{itemdescr}



\rSec2[internet.resolver.results.assign]{\tcode{ip::basic_resolver_results} assignment}

\begin{itemdecl}
basic_resolver_results& operator=(const basic_resolver_results& rhs);
\end{itemdecl}

\begin{itemdescr}
\pnum
\postconditions \tcode{*this == rhs}.

\pnum
\returns \tcode{*this}.
\end{itemdescr}

\begin{itemdecl}
basic_resolver_results& operator=(basic_resolver_results& rhs) noexcept;
\end{itemdecl}

\begin{itemdescr}
\pnum
\postconditions \tcode{*this} is equal to the prior value of \tcode{rhs}.

\pnum
\returns \tcode{*this}.
\end{itemdescr}



\rSec2[internet.resolver.results.size]{\tcode{ip::basic_resolver_results} size}

\begin{itemdecl}
size_type size() const noexcept;
\end{itemdecl}

\begin{itemdescr}
\pnum
\returns The number of \tcode{basic_resolver_entry} elements in \tcode{*this}.
\end{itemdescr}

\begin{itemdecl}
size_type max_size() const noexcept;
\end{itemdecl}

\begin{itemdescr}
\pnum
\returns The maximum number of \tcode{basic_resolver_entry} elements that can be stored in \tcode{*this}.
\end{itemdescr}

\begin{itemdecl}
bool empty() const noexcept;
\end{itemdecl}

\begin{itemdescr}
\pnum
\returns \tcode{size() == 0}.
\end{itemdescr}



\rSec2[internet.resolver.results.access]{\tcode{ip::basic_resolver_results} element access}

\begin{itemdecl}
const_iterator begin() const;
const_iterator cbegin() const;
\end{itemdecl}

\begin{itemdescr}
\pnum
\returns A starting iterator that enumerates over all the \tcode{basic_resolver_entry} elements stored in \tcode{*this}.
\end{itemdescr}

\begin{itemdecl}
const_iterator end() const;
const_iterator cend() const;
\end{itemdecl}

\begin{itemdescr}
\pnum
\returns A terminating iterator that enumerates over all the \tcode{basic_resolver_entry} elements stored in \tcode{*this}.
\end{itemdescr}



\rSec2[internet.resolver.results.swap]{\tcode{ip::basic_resolver_results} swap}

\begin{itemdecl}
void swap(basic_resolver_results& that) noexcept;
\end{itemdecl}

\begin{itemdescr}
\pnum
\postconditions \tcode{*this} is equal to the prior value of \tcode{that}, and \tcode{that} is equal to the prior value of \tcode{*this}.
\end{itemdescr}



\rSec2[internet.resolver.results.comparisons]{\tcode{ip::basic_resolver_results} comparisons}

\begin{itemdecl}
template<class InternetProtocol>
  bool operator==(const basic_resolver_results<InternetProtocol>& a,
                  const basic_resolver_results<InternetProtocol>& b);
\end{itemdecl}

\begin{itemdescr}
\pnum
\returns \tcode{a.size() == b.size() \&\& equal(a.cbegin(), a.cend(), b.cbegin())}.
\end{itemdescr}

\begin{itemdecl}
template<class InternetProtocol>
  bool operator!=(const basic_resolver_results<InternetProtocol>& a,
                  const basic_resolver_results<InternetProtocol>& b);
\end{itemdecl}

\begin{itemdescr}
\pnum
\returns \tcode{!(a == b)}.
\end{itemdescr}




\indexlibrary{\idxcode{resolver_base}}%
\rSec1[internet.resolver.base]{Class \tcode{ip::resolver_base}}

\indexlibrary{\idxcode{resolve_base}}%
\begin{codeblock}
namespace std {
namespace experimental {
namespace net {
inline namespace v1 {
namespace ip {

  class resolver_base
  {
  public:
    using flags = @\textit{T1}@;
    static const flags passive;
    static const flags canonical_name;
    static const flags numeric_host;
    static const flags numeric_service;
    static const flags v4_mapped;
    static const flags all_matching;
    static const flags address_configured;

  protected:
    resolver_base();
    ~resolver_base();
  };

} // namespace ip
} // inline namespace v1
} // namespace net
} // namespace experimental
} // namespace std
\end{codeblock}

\pnum
\tcode{resolver_base} defines a bitmask type, \tcode{flags}, with the bitmask elements shown in Table~\ref{tab:internet.resolver.base.requirements}.

\begin{libreqtab3}
{Resolver flags}
{tab:internet.resolver.base.requirements}
\\ \topline
\lhdr{Constant name}  &
\chdr{POSIX macro}  &
\rhdr{Definition or notes}  \\ \capsep
\endfirsthead
\continuedcaption\\
\hline
\lhdr{Constant name}  &
\chdr{POSIX macro}  &
\rhdr{Definition or notes}  \\ \capsep
\endhead

\tcode{passive}  &
\tcode{AI_PASSIVE}  &
 Returned endpoints are intended for use as locally bound socket endpoints.  \\ \rowsep

\tcode{canonical_name}  &
\tcode{AI_CANONNAME}  &
 Determine the canonical name of the host specified in the query.  \\ \rowsep

\tcode{numeric_host}  &
\tcode{AI_NUMERICHOST}  &
 Host name should be treated as a numeric string defining an IPv4 or IPv6 address and no host name resolution should be attempted.  \\ \rowsep

\tcode{numeric_service}  &
\tcode{AI_NUMERICSERV}  &
 Service name should be treated as a numeric string defining a port number and no service name resolution should be attempted.  \\ \rowsep

\tcode{v4_mapped}  &
\tcode{AI_V4MAPPED}  &
 If the protocol is specified as an IPv6 protocol, return IPv4-mapped IPv6 addresses on finding no IPv6 addresses.  \\ \rowsep

\tcode{all_matching}  &
\tcode{AI_ALL}  &
If used with \tcode{v4_mapped}, return all matching IPv6 and IPv4 addresses.  \\ \rowsep

\tcode{address_configured}  &
\tcode{AI_ADDRCONFIG}  &
 Only return IPv4 addresses if a non-loopback IPv4 address is configured for the system. Only return IPv6 addresses if a non-loopback IPv6 address is configured for the system.  \\

\end{libreqtab3}



\indexlibrary{\idxcode{basic_resolver}}%
\rSec1[internet.resolver]{Class template \tcode{ip::basic_resolver}}

\pnum
Objects of type \tcode{basic_resolver<InternetProtocol>} are used to perform name resolution. Name resolution is the translation of a host name and service name into a sequence of endpoints, or the translation of an endpoint into its corresponding host name and service name.

\begin{codeblock}
namespace std {
namespace experimental {
namespace net {
inline namespace v1 {
namespace ip {

  template<class InternetProtocol>
  class basic_resolver : public resolver_base
  {
  public:
    // types:

    using executor_type = io_context::executor_type;
    using protocol_type = InternetProtocol;
    using endpoint_type = typename InternetProtocol::endpoint;
    using results_type = basic_resolver_results<InternetProtocol>;

    // \ref{internet.resolver.cons}, construct / copy / destroy:

    explicit basic_resolver(io_context& ctx);
    basic_resolver(const basic_resolver&) = delete;
    basic_resolver(basic_resolver&& rhs) noexcept;

    ~basic_resolver();

    basic_resolver& operator=(const basic_resolver&) = delete;
    basic_resolver& operator=(basic_resolver&& rhs);

    // \ref{internet.resolver.ops}, basic_resolver operations:

    executor_type get_executor() noexcept;

    void cancel();

    results_type resolve(string_view host_name, string_view service_name);
    results_type resolve(string_view host_name, string_view service_name,
                         error_code& ec);
    results_type resolve(string_view host_name, string_view service_name,
                         flags f);
    results_type resolve(string_view host_name, string_view service_name,
                         flags f, error_code& ec);

    template<class CompletionToken>
      @\DEDUCED@ async_resolve(string_view host_name, string_view service_name,
                            CompletionToken&& token);
    template<class CompletionToken>
      @\DEDUCED@ async_resolve(string_view host_name, string_view service_name,
                            flags f, CompletionToken&& token);

    results_type resolve(const protocol_type& protocol,
                         string_view host_name, string_view service_name);
    results_type resolve(const protocol_type& protocol,
                         string_view host_name, string_view service_name,
                         error_code& ec);
    results_type resolve(const protocol_type& protocol,
                         string_view host_name, string_view service_name,
                         flags f);
    results_type resolve(const protocol_type& protocol,
                         string_view host_name, string_view service_name,
                         flags f, error_code& ec);

    template<class CompletionToken>
      @\DEDUCED@ async_resolve(const protocol_type& protocol,
                            string_view host_name, string_view service_name,
                            CompletionToken&& token);
    template<class CompletionToken>
      @\DEDUCED@ async_resolve(const protocol_type& protocol,
                            string_view host_name, string_view service_name,
                            flags f, CompletionToken&& token);

    results_type resolve(const endpoint_type& e);
    results_type resolve(const endpoint_type& e, error_code& ec);

    template<class CompletionToken>
      @\DEDUCED@ async_resolve(const endpoint_type& e,
                            CompletionToken&& token);
  };

} // namespace ip
} // inline namespace v1
} // namespace net
} // namespace experimental
} // namespace std
\end{codeblock}


\rSec2[internet.resolver.cons]{\tcode{ip::basic_resolver} constructors}

\begin{itemdecl}
explicit basic_resolver(io_context& ctx);
\end{itemdecl}

\begin{itemdescr}
\pnum
\postconditions \tcode{get_executor() == ctx.get_executor()}.
\end{itemdescr}

\begin{itemdecl}
basic_resolver(basic_resolver&& rhs) noexcept;
\end{itemdecl}

\begin{itemdescr}
\pnum
\effects Move constructs an object of class \tcode{basic_resolver<InternetProtocol>} that refers to the state originally represented by \tcode{rhs}.

\pnum
\postconditions \tcode{get_executor() == rhs.get_executor()}.
\end{itemdescr}



\rSec2[internet.resolver.dtor]{\tcode{ip::basic_resolver} destructor}

\begin{itemdecl}
~basic_resolver();
\end{itemdecl}

\begin{itemdescr}
\pnum
\effects Destroys the resolver, canceling all asynchronous operations associated with this resolver as if by calling \tcode{cancel()}.
\end{itemdescr}



\rSec2[internet.resolver.assign]{\tcode{ip::basic_resolver} assignment}

\begin{itemdecl}
basic_resolver& operator=(basic_resolver&& rhs);
\end{itemdecl}

\begin{itemdescr}
\pnum
\effects Cancels all outstanding asynchronous operations associated with \tcode{*this} as if by calling \tcode{cancel()}, then moves into \tcode{*this} the state originally represented by \tcode{rhs}.

\pnum
\postconditions \tcode{get_executor() == ctx.get_executor()}.

\pnum
\returns \tcode{*this}.
\end{itemdescr}



\rSec2[internet.resolver.ops]{\tcode{ip::basic_resolver} operations}

\begin{itemdecl}
executor_type get_executor() noexcept;
\end{itemdecl}

\begin{itemdescr}
\pnum
\returns The associated executor.
\end{itemdescr}

\begin{itemdecl}
void cancel();
\end{itemdecl}

\begin{itemdescr}
\pnum
\effects Cancels all outstanding asynchronous resolve operations associated with \tcode{*this}. Completion handlers for canceled operations are passed an error code \tcode{ec} such that \tcode{ec == errc::operation_canceled} yields \tcode{true}.

\pnum
\remarks Does not block (\CppXref{defns.block}) the calling thread pending completion of the canceled operations.
\end{itemdescr}

\begin{itemdecl}
results_type resolve(string_view host_name, string_view service_name);
results_type resolve(string_view host_name, string_view service_name,
                     error_code& ec);
\end{itemdecl}

\begin{itemdescr}
\pnum
\returns \tcode{resolve(host_name, service_name, resolver_base::flags(), ec)}.
\end{itemdescr}

\begin{itemdecl}
results_type resolve(string_view host_name, string_view service_name,
                     flags f);
results_type resolve(string_view host_name, string_view service_name,
                     flags f, error_code& ec);
\end{itemdecl}

\begin{itemdescr}
\pnum
\effects If \tcode{host_name.data() != nullptr}, let H be an \ntbs constructed from \tcode{host_name}; otherwise, let H be \tcode{nullptr}. If \tcode{service_name.data() != nullptr}, let S be an \ntbs constructed from \tcode{service_name}; otherwise, let S be \tcode{nullptr}. Resolves a host name and service name, as if by POSIX:
\begin{codeblock}
addrinfo hints;
hints.ai_flags = static_cast<int>(f);
hints.ai_family = AF_UNSPEC;
hints.ai_socktype = endpoint_type().protocol().type();
hints.ai_protocol = endpoint_type().protocol().protocol();
hints.ai_addr = nullptr;
hints.ai_addrlen = 0;
hints.ai_canonname = nullptr;
hints.ai_next = nullptr;
addrinfo* result = nullptr;
getaddrinfo(H, S, &hints, &result);
\end{codeblock}


\pnum
\returns On success, a non-empty results object containing the results of the resolve operation. Otherwise \tcode{results_type()}.
\end{itemdescr}

\begin{itemdecl}
template<class CompletionToken>
  @\DEDUCED@ async_resolve(string_view host_name, string_view service_name,
                        CompletionToken&& token);
\end{itemdecl}

\begin{itemdescr}
\pnum
\returns \tcode{async_resolve(host_name, service_name, resolver_base::flags(), forward<Comp\-l\-et\-ion\-Token>(token))}.
\end{itemdescr}

\begin{itemdecl}
template<class CompletionToken>
  @\DEDUCED@ async_resolve(string_view host_name, string_view service_name,
                        flags f, CompletionToken&& token);
\end{itemdecl}

\begin{itemdescr}
\pnum
\completionsig \tcode{void(error_code ec, results_type r)}.

\pnum
\effects If \tcode{host_name.data() != nullptr}, let H be an \ntbs constructed from \tcode{host_name}; otherwise, let H be \tcode{nullptr}. If \tcode{service_name.data() != nullptr}, let S be an \ntbs constructed from \tcode{service_name}; otherwise, let S be \tcode{nullptr}. Initiates an asynchronous operation to resolve a host name and service name, as if by POSIX:
\begin{codeblock}
addrinfo hints;
hints.ai_flags = static_cast<int>(f);
hints.ai_family = AF_UNSPEC;
hints.ai_socktype = endpoint_type().protocol().type();
hints.ai_protocol = endpoint_type().protocol().protocol();
hints.ai_addr = nullptr;
hints.ai_addrlen = 0;
hints.ai_canonname = nullptr;
hints.ai_next = nullptr;
addrinfo* result = nullptr;
getaddrinfo(H, S, &hints, &result);
\end{codeblock}
 On success, \tcode{r} is a non-empty results object containing the results of the resolve operation. Otherwise, \tcode{r} is \tcode{results_type()}.
\end{itemdescr}

\begin{itemdecl}
results_type resolve(const protocol_type& protocol,
                     string_view host_name, string_view service_name);
results_type resolve(const protocol_type& protocol,
                     string_view host_name, string_view service_name,
                     error_code& ec);
\end{itemdecl}

\begin{itemdescr}
\pnum
\returns \tcode{resolve(protocol, host_name, service_name, resolver_base::flags(), ec)}.
\end{itemdescr}

\begin{itemdecl}
results_type resolve(const protocol_type& protocol,
                     string_view host_name, string_view service_name,
                     flags f);
results_type resolve(const protocol_type& protocol,
                     string_view host_name, string_view service_name,
                     flags f, error_code& ec);
\end{itemdecl}

\begin{itemdescr}
\pnum
\effects If \tcode{host_name.data() != nullptr}, let H be an \ntbs constructed from \tcode{host_name}; otherwise, let H be \tcode{nullptr}. If \tcode{service_name.data() != nullptr}, let S be an \ntbs constructed from \tcode{service_name}; otherwise, let S be \tcode{nullptr}. Resolves a host name and service name, as if by POSIX:
\begin{codeblock}
addrinfo hints;
hints.ai_flags = static_cast<int>(f);
hints.ai_family = protocol.family();
hints.ai_socktype = protocol.type();
hints.ai_protocol = protocol.protocol();
hints.ai_addr = nullptr;
hints.ai_addrlen = 0;
hints.ai_canonname = nullptr;
hints.ai_next = nullptr;
addrinfo* result = nullptr;
getaddrinfo(H, S, &hints, &result);
\end{codeblock}


\pnum
\returns On success, a non-empty results object containing the results of the resolve operation. Otherwise \tcode{results_type()}.
\end{itemdescr}

\begin{itemdecl}
template<class CompletionToken>
  @\DEDUCED@ async_resolve(const protocol_type& protocol,
                        string_view host_name, string_view service_name,
                        CompletionToken&& token);
\end{itemdecl}

\begin{itemdescr}
\pnum
\returns \tcode{async_resolve(protocol, host_name, service_name, resolver_base::flags(), for\-ward<CompletionToken>(token))}.
\end{itemdescr}

\begin{itemdecl}
template<class CompletionToken>
  @\DEDUCED@ async_resolve(const protocol& protocol,
                        string_view host_name, string_view service_name,
                        flags f, CompletionToken&& token);
\end{itemdecl}

\begin{itemdescr}
\pnum
\completionsig \tcode{void(error_code ec, results_type r)}.

\pnum
\effects If \tcode{host_name.data() != nullptr}, let H be an \ntbs constructed from \tcode{host_name}; otherwise, let H be \tcode{nullptr}. If \tcode{service_name.data() != nullptr}, let S be an \ntbs constructed from \tcode{service_name}; otherwise, let S be \tcode{nullptr}. Initiates an asynchronous operation to resolve a host name and service name, as if by POSIX:
\begin{codeblock}
addrinfo hints;
hints.ai_flags = static_cast<int>(f);
hints.ai_family = protocol.family();
hints.ai_socktype = protocol.type();
hints.ai_protocol = protocol.protocol();
hints.ai_addr = nullptr;
hints.ai_addrlen = 0;
hints.ai_canonname = nullptr;
hints.ai_next = nullptr;
addrinfo* result = nullptr;
getaddrinfo(H, S, &hints, &result);
\end{codeblock}
 On success, \tcode{r} is a non-empty results object containing the results of the resolve operation. Otherwise, \tcode{r} is \tcode{results_type()}.
\end{itemdescr}

\begin{itemdecl}
results_type resolve(const endpoint_type& e);
results_type resolve(const endpoint_type& e, error_code& ec);
\end{itemdecl}

\begin{itemdescr}
\pnum
\effects Let \tcode{S1} and \tcode{S2} be \impldef{maximum length of host and service names} values that are sufficiently large to hold the host name and service name respectively. Resolves an endpoint as if by POSIX:
\begin{codeblock}
char host_name[S1];
char service_name[S2];
int flags = 0;
if (endpoint_type().protocol().type() == SOCK_DGRAM)
  flags |= NI_DGRAM;
int result = getnameinfo((const sockaddr*)e.data(), e.size(),
                         host_name, S1,
                         service_name, S2,
                         flags);
if (result != 0)
{
  flags |= NI_NUMERICSERV;
  result = getnameinfo((const sockaddr*)e.data(), e.size(),
                       host_name, S1,
                       service_name, S2,
                       flags);
}
\end{codeblock}


\pnum
\returns On success, a results object with \tcode{size() == 1} containing the results of the resolve operation. Otherwise \tcode{results_type()}.
\end{itemdescr}

\begin{itemdecl}
template<class CompletionToken>
  @\DEDUCED@ async_resolve(const endpoint_type& e,
                        CompletionToken&& token);
\end{itemdecl}

\begin{itemdescr}
\pnum
\completionsig \tcode{void(error_code ec, results_type r)}.

\pnum
\effects Let \tcode{S1} and \tcode{S2} be \impldef{maximum length of host and service names} values that are sufficiently large to hold the host name and service name respectively. Initiates an asynchronous operation to resolve an endpoint as if by POSIX:
\begin{codeblock}
char host_name[S1];
char service_name[S2];
int flags = 0;
if (endpoint_type().protocol().type() == SOCK_DGRAM)
  flags |= NI_DGRAM;
int result = getnameinfo((const sockaddr*)e.data(), e.size(),
                         host_name, S1,
                         service_name, S2,
                         flags);
if (result != 0)
{
  flags |= NI_NUMERICSERV;
  result = getnameinfo((const sockaddr*)e.data(), e.size(),
                       host_name, S1,
                       service_name, S2,
                       flags);
}
\end{codeblock}
 On success, \tcode{r} is a results object with \tcode{size() == 1} containing the results of the resolve operation; otherwise, \tcode{r} is \tcode{results_type()}.
\end{itemdescr}




\rSec1[internet.host.name]{Host name functions}

\indexlibrary{\idxcode{host_name}}%
\begin{itemdecl}
string host_name();
string host_name(error_code& ec);
template<class Allocator>
  basic_string<char, char_traits<char>, Allocator>
    host_name(const Allocator& a);
template<class Allocator>
  basic_string<char, char_traits<char>, Allocator>
    host_name(const Allocator& a, error_code& ec);
\end{itemdecl}

\begin{itemdescr}
\pnum
\returns The standard host name for the current machine, determined as if by POSIX \tcode{gethostname}.

\pnum
\remarks In the last two overloads, ill-formed unless \tcode{allocator_traits<Allocator>::value_type} is \tcode{char}.
\end{itemdescr}



\indexlibrary{\idxcode{tcp}}%
\rSec1[internet.tcp]{Class \tcode{ip::tcp}}

\pnum
The class \tcode{tcp} encapsulates the types and flags necessary for TCP sockets.

\begin{codeblock}
namespace std {
namespace experimental {
namespace net {
inline namespace v1 {
namespace ip {

  class tcp
  {
  public:
    // types:
    using endpoint = basic_endpoint<tcp>;
    using resolver = basic_resolver<tcp>;
    using socket = basic_stream_socket<tcp>;
    using acceptor = basic_socket_acceptor<tcp>;
    using iostream = basic_socket_iostream<tcp>;
    class no_delay;

    // static members:
    static constexpr tcp v4() noexcept;
    static constexpr tcp v6() noexcept;

    tcp() = delete;
  };

  // \ref{internet.tcp.comparisons}, tcp comparisons:
  constexpr bool operator==(const tcp& a, const tcp& b) noexcept;
  constexpr bool operator!=(const tcp& a, const tcp& b) noexcept;

} // namespace ip
} // inline namespace v1
} // namespace net
} // namespace experimental
} // namespace std
\end{codeblock}

\pnum
The \tcode{tcp} class meets the requirements for an \tcode{InternetProtocol}~(\ref{internet.reqmts.protocol}).

\indexextensible{tcp}%
\pnum
 Extensible implementations provide the following member functions:

\begin{codeblock}
namespace std {
namespace experimental {
namespace net {
inline namespace v1 {
namespace ip {

  class tcp
  {
  public:
    constexpr int family() const noexcept;
    constexpr int type() const noexcept;
    constexpr int protocol() const noexcept;
    @\textit{// remainder unchanged}@
  };

} // namespace ip
} // inline namespace v1
} // namespace net
} // namespace experimental
} // namespace std
\end{codeblock}

\pnum
 The return values for these member functions are listed in Table~\ref{tab:internet.tcp.requirements}.

\begin{libreqtab4}
{Behavior of extensible \tcode{ip::tcp} implementations}
{tab:internet.tcp.requirements}
\\ \topline
\lhdr{value}  &
\chdr{family()}  &
\chdr{type()}  &
\rhdr{protocol()}  \\ \capsep
\endfirsthead
\continuedcaption\\
\hline
\lhdr{value}  &
\chdr{family()}  &
\chdr{type()}  &
\rhdr{protocol()}  \\ \capsep
\endhead

\tcode{tcp::v4()}  &
\tcode{AF_INET}  &
\tcode{SOCK_STREAM}  &
\tcode{IPPROTO_TCP}  \\ \rowsep

\tcode{tcp::v6()}  &
\tcode{AF_INET6}  &
\tcode{SOCK_STREAM}  &
\tcode{IPPROTO_TCP}  \\

\end{libreqtab4}

\pnum
\begin{note} The constants \tcode{AF_INET}, \tcode{AF_INET6} and \tcode{SOCK_STREAM} are defined in the POSIX \tcode{<sys/socket.h>} header. The constant \tcode{IPPROTO_TCP} is defined in the POSIX \tcode{<netinet/in.h>} header. \end{note}


\rSec2[internet.tcp.comparisons]{\tcode{ip::tcp} comparisons}

\begin{itemdecl}
constexpr bool operator==(const tcp& a, const tcp& b) noexcept;
\end{itemdecl}

\begin{itemdescr}
\pnum
\returns A boolean indicating whether two objects of class \tcode{tcp} are equal, such that the expression \tcode{tcp::v4() == tcp::v4()} is \tcode{true}, the expression \tcode{tcp::v6() == tcp::v6()} is \tcode{true}, and the expression \tcode{tcp::v4() == tcp::v6()} is \tcode{false}.
\end{itemdescr}

\begin{itemdecl}
constexpr bool operator!=(const tcp& a, const tcp& b) noexcept;
\end{itemdecl}

\begin{itemdescr}
\pnum
\returns \tcode{!(a == b)}.
\end{itemdescr}




\indexlibrary{\idxcode{udp}}%
\rSec1[internet.udp]{Class \tcode{ip::udp}}

\pnum
The class \tcode{udp} encapsulates the types and flags necessary for UDP sockets.

\begin{codeblock}
namespace std {
namespace experimental {
namespace net {
inline namespace v1 {
namespace ip {

  class udp
  {
  public:
    // types:
    using endpoint = basic_endpoint<udp>;
    using resolver = basic_resolver<udp>;
    using socket = basic_datagram_socket<udp>;

    // static members:
    static constexpr udp v4() noexcept;
    static constexpr udp v6() noexcept;

    udp() = delete;
  };

  // \ref{internet.udp.comparisons}, udp comparisons:
  constexpr bool operator==(const udp& a, const udp& b) noexcept;
  constexpr bool operator!=(const udp& a, const udp& b) noexcept;

} // namespace ip
} // inline namespace v1
} // namespace net
} // namespace experimental
} // namespace std
\end{codeblock}

\pnum
The \tcode{udp} class meets the requirements for an \tcode{InternetProtocol}~(\ref{internet.reqmts.protocol}).

\indexextensible{udp}%
\pnum
 Extensible implementations provide the following member functions:

\begin{codeblock}
namespace std {
namespace experimental {
namespace net {
inline namespace v1 {
namespace ip {

  class udp
  {
  public:
    constexpr int family() const noexcept;
    constexpr int type() const noexcept;
    constexpr int protocol() const noexcept;
    @\textit{// remainder unchanged}@
  };

} // namespace ip
} // inline namespace v1
} // namespace net
} // namespace experimental
} // namespace std
\end{codeblock}

\pnum
 The return values for these member functions are listed in Table~\ref{tab:internet.udp.requirements}.

\begin{libreqtab4}
{Behavior of extensible \tcode{ip::udp} implementations}
{tab:internet.udp.requirements}
\\ \topline
\lhdr{value}  &
\chdr{family()}  &
\chdr{type()}  &
\rhdr{protocol()}  \\ \capsep
\endfirsthead
\continuedcaption\\
\hline
\lhdr{value}  &
\chdr{family()}  &
\chdr{type()}  &
\rhdr{protocol()}  \\ \capsep
\endhead

\tcode{udp::v4()}  &
\tcode{AF_INET}  &
\tcode{SOCK_DGRAM}  &
\tcode{IPPROTO_UDP}  \\ \rowsep

\tcode{udp::v6()}  &
\tcode{AF_INET6}  &
\tcode{SOCK_DGRAM}  &
\tcode{IPPROTO_UDP}  \\

\end{libreqtab4}

\pnum
\begin{note} The constants \tcode{AF_INET}, \tcode{AF_INET6} and \tcode{SOCK_DGRAM} are defined in the POSIX \tcode{<sys/socket.h>} header. The constant \tcode{IPPROTO_UDP} is defined in the POSIX \tcode{<netinet/in.h>} header. \end{note}


\rSec2[internet.udp.comparisons]{\tcode{ip::udp} comparisons}

\begin{itemdecl}
constexpr bool operator==(const udp& a, const udp& b) noexcept;
\end{itemdecl}

\begin{itemdescr}
\pnum
\returns A boolean indicating whether two objects of class \tcode{udp} are equal, such that the expression \tcode{udp::v4() == udp::v4()} is \tcode{true}, the expression \tcode{udp::v6() == udp::v6()} is \tcode{true}, and the expression \tcode{udp::v4() == udp::v6()} is \tcode{false}.
\end{itemdescr}

\begin{itemdecl}
constexpr bool operator!=(const udp& a, const udp& b) noexcept;
\end{itemdecl}

\begin{itemdescr}
\pnum
\returns \tcode{!(a == b)}.
\end{itemdescr}




\rSec1[internet.socket.opt]{Internet socket options}

\pnum
In Table~\ref{tab:internet.socket.opt.requirements}, let \placeholder{C} denote a socket option class; let \placeholder{L} identify the POSIX macro to be passed as the \tcode{level} argument to POSIX \tcode{setsockopt} and getsockopt; let \placeholder{N} identify the POSIX macro to be passed as the \tcode{option_name} argument to POSIX \tcode{setsockopt} and \tcode{getsockopt}; let \placeholder{T} identify the type of the value whose address will be passed as the \tcode{option_value} argument to POSIX \tcode{setsockopt} and \tcode{getsockopt};
let \tcode{p} denote a value of a (possibly const) type meeting the protocol~(\ref{socket.reqmts.protocol}) requirements,
as passed to the socket option's \tcode{level} and \tcode{name} member functions;
and let \tcode{F} be the value of \tcode{p.family()}.

\begin{LongTable}
{Internet socket options}
{tab:internet.socket.opt.requirements}
{x{.15\hsize}x{.18\hsize}x{.17\hsize}x{.17\hsize}x{.21\hsize}}
\\ \topline
\lhdr{\placeholder{C}}  &
\chdr{\placeholder{L}}  &
\chdr{\placeholder{N}}  &
\chdr{\placeholder{T}}  &
\rhdr{Requirements,}  \\
  &
  &
  &
  &
\rhdr{definition or notes}  \\ \capsep
\endfirsthead
\continuedcaption\\
\hline
\lhdr{\placeholder{C}}  &
\chdr{\placeholder{L}}  &
\chdr{\placeholder{N}}  &
\chdr{\placeholder{T}}  &
\rhdr{Requirements,}  \\
  &
  &
  &
  &
\rhdr{definition or notes}  \\ \capsep
\endhead

\tcode{ip::tcp::}\br
\tcode{no_delay}  &
\tcode{IPPROTO_TCP}  &
\tcode{TCP_NODELAY}  &
\tcode{int}  &
Satisfies the \tcode{BooleanSocket\-Option}~(\ref{socket.reqmts.opt.bool}) type requirements. Determines whether a TCP socket will avoid coalescing of small segments. \begin{note} That is, setting this option disables the Nagle algorithm. \end{note}  \\ \rowsep

\tcode{ip::v6_only}  &
\tcode{IPPROTO_IPV6}  &
\tcode{IPV6_V6ONLY}  &
\tcode{int}  &
Satisfies the \tcode{BooleanSocket\-Option}~(\ref{socket.reqmts.opt.bool}) type requirements. Determines whether a socket created for an IPv6 protocol is restricted to IPv6 communications only. Implementations are not required to support setting the \tcode{v6_only} option to \tcode{false}, and the initial value of the \tcode{v6_only} option for a socket is \impldef{initial value of the \tcode{v6_only} option for a socket}. \begin{note} As not all operating systems support dual stack IP networking. Some operating systems that do provide dual stack support offer a configuration option to disable it or to set the initial value of the \tcode{v6_only} socket option. \end{note}  \\ \rowsep

\tcode{ip::unicast::}
\tcode{hops}  &
\tcode{IPPROTO_IPV6} if \tcode{F == AF_INET6}, otherwise \tcode{IPPROTO_IP}  &
\tcode{IPV6_UNICAST_HOPS} if \tcode{F == AF_INET6}, otherwise \tcode{IP_TTL}  &
\tcode{int}  &
Satisfies the \tcode{IntegerSocket\-Option}~(\ref{socket.reqmts.opt.int}) type requirements. Specifies the default number of hops (also known as time-to-live or TTL) on outbound datagrams. The constructor and assignment operator for the \tcode{ip::unicast::hops} class throw \tcode{out_of_range} if the \tcode{int} argument is not in the range \crange{0}{255}.  \\ \rowsep

\tcode{ip::multicast::}
\tcode{join_group}  &
\tcode{IPPROTO_IPV6} if \tcode{F == AF_INET6}, otherwise \tcode{IPPROTO_IP}  &
\tcode{IPV6_JOIN_GROUP} if \tcode{F == AF_INET6}, otherwise \tcode{IP_ADD_MEMBERSHIP}  &
\tcode{ipv6_mreq} if \tcode{F == AF_INET6}, otherwise \tcode{ip_mreq}  &
Satisfies the \tcode{MulticastGroupSock\-etOption}~(\ref{internet.reqmts.opt.mcast}) type requirements. Requests that the socket join the specified multicast group.  \\ \rowsep

\tcode{ip::multicast::}
\tcode{leave_group}  &
\tcode{IPPROTO_IPV6} if \tcode{F == AF_INET6}, otherwise \tcode{IPPROTO_IP}  &
\tcode{IPV6_LEAVE_GROUP} if \tcode{F == AF_INET6}, otherwise \tcode{IP_DROP_MEMBERSHIP}  &
\tcode{ipv6_mreq} if \tcode{F == AF_INET6}, otherwise \tcode{ip_mreq}  &
Satisfies the \tcode{MulticastGroupSock\-etOption}~(\ref{internet.reqmts.opt.mcast}) type requirements. Requests that the socket leave the specified multicast group.  \\ \rowsep

\tcode{ip::multicast::}
\tcode{outbound_interface}\br~(\ref{internet.multicast.outbound})  &
\tcode{IPPROTO_IPV6} if \tcode{F == AF_INET6}, otherwise \tcode{IPPROTO_IP}  &
\tcode{IPV6_MULTICAST_IF} if \tcode{F == AF_INET6}, otherwise \tcode{IP_MULTICAST_IF}  &
\tcode{unsigned int} if \tcode{F == AF_INET6}, otherwise \tcode{in_addr}  &
 Specifies the network interface to use for outgoing multicast datagrams.  \\ \rowsep

\tcode{ip::multicast::}
\tcode{hops}  &
\tcode{IPPROTO_IPV6} if \tcode{F == AF_INET6}, otherwise \tcode{IPPROTO_IP}  &
\tcode{IPV6_MULTICAST_HOPS} if \tcode{F == AF_INET6}, otherwise \tcode{IP_MULTICAST_TTL}  &
\tcode{int}  &
Satisfies the \tcode{IntegerSocket\-Option}~(\ref{socket.reqmts.opt.int}) type requirements. Specifies the default number of hops (also known as time-to-live or TTL) on outbound datagrams. The constructor and assignment operator for the \tcode{ip::multicast::hops} class throw \tcode{out_of_range} if the \tcode{int} argument is not in the range \crange{0}{255}.  \\ \rowsep

\tcode{ip::multicast::}
\tcode{enable_loopback}  &
\tcode{IPPROTO_IPV6} if \tcode{F == AF_INET6}, otherwise \tcode{IPPROTO_IP}  &
\tcode{IPV6_MULTICAST_LOOP} if \tcode{F == AF_INET6}, otherwise \tcode{IP_MULTICAST_LOOP}  &
\tcode{int}  &
Satisfies the \tcode{BooleanSocket\-Option}~(\ref{socket.reqmts.opt.bool}) type requirements. Determines whether multicast datagrams are delivered back to the local application.  \\

\end{LongTable}


\indexlibrary{\idxcode{outbound_interface}}%
\rSec2[internet.multicast.outbound]{Class \tcode{ip::multicast::outbound_interface}}

\pnum
The \tcode{outbound_interface} class represents a socket option that specifies the network interface to use for outgoing multicast datagrams.

\begin{codeblock}
namespace std {
namespace experimental {
namespace net {
inline namespace v1 {
namespace ip {
namespace multicast {

  class outbound_interface
  {
  public:
    // constructors:
    explicit outbound_interface(const address_v4& network_interface) noexcept;
    explicit outbound_interface(unsigned int network_interface) noexcept;
  };

} // namespace multicast
} // namespace ip
} // inline namespace v1
} // namespace net
} // namespace experimental
} // namespace std
\end{codeblock}

\pnum
\tcode{outbound_interface} satisfies the requirements for \tcode{Destructible} (\CppXref{destructible}), \tcode{CopyConstructible} (\CppXref{copyconstructible}), \tcode{CopyAssignable} (\CppXref{copyassignable}), and \tcode{SettableSocketOption}~(\ref{socket.reqmts.settablesocketoption}).

\indexextensible{outbound_interface}%
\pnum
 Extensible implementations provide the following member functions:

\begin{codeblock}
namespace std {
namespace experimental {
namespace net {
inline namespace v1 {
namespace ip {
namespace multicast {

  class outbound_interface
  {
  public:
    template<class Protocol> int level(const Protocol& p) const noexcept;
    template<class Protocol> int name(const Protocol& p) const noexcept;
    template<class Protocol> const void* data(const Protocol& p) const noexcept;
    template<class Protocol> size_t size(const Protocol& p) const noexcept;
    // \textit{remainder unchanged}
  private:
      in_addr v4_value_; // \expos
      unsigned int v6_value_; // \expos
  };

} // namespace multicast
} // namespace ip
} // inline namespace v1
} // namespace net
} // namespace experimental
} // namespace std
\end{codeblock}

\begin{itemdecl}
explicit outbound_interface(const address_v4& network_interface) noexcept;
\end{itemdecl}

\begin{itemdescr}
\pnum
\effects For extensible implementations, \tcode{v4_value_} is initialized to correspond to the IPv4 address \tcode{network_interface}, and \tcode{v6_value_} is zero-initialized.
\end{itemdescr}

\begin{itemdecl}
explicit outbound_interface(unsigned int network_interface) noexcept;
\end{itemdecl}

\begin{itemdescr}
\pnum
\effects For extensible implementations, \tcode{v6_value_} is initialized to \tcode{network_interface}, and \tcode{v4_value_} is zero-initialized.
\end{itemdescr}

\begin{itemdecl}
template<class Protocol> int level(const Protocol& p) const noexcept;
\end{itemdecl}

\begin{itemdescr}
\pnum
\returns \tcode{IPPROTO_IPV6} if \tcode{p.family() == AF_INET6}, otherwise \tcode{IPPROTO_IP}.
\end{itemdescr}

\begin{itemdecl}
template<class Protocol> int name(const Protocol& p) const noexcept;
\end{itemdecl}

\begin{itemdescr}
\pnum
\returns \tcode{IPV6_MULTICAST_IF} if \tcode{p.family() == AF_INET6}, otherwise \tcode{IP_MULTICAST_IF}.
\end{itemdescr}

\begin{itemdecl}
template<class Protocol> const void* data(const Protocol& p) const noexcept;
\end{itemdecl}

\begin{itemdescr}
\pnum
\returns \tcode{addressof(v6_value_)} if \tcode{p.family() == AF_INET6}, otherwise \tcode{addressof(v4_value_)}.
\end{itemdescr}

\begin{itemdecl}
template<class Protocol> size_t size(const Protocol& p) const noexcept;
\end{itemdecl}

\begin{itemdescr}
\pnum
\returns \tcode{sizeof(v6_value_)} if \tcode{p.family() == AF_INET6}, otherwise \tcode{sizeof(v4_value_)}.
\end{itemdescr}




