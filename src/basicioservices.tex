%!TEX root = ts.tex

\rSec0[io_context]{Basic I/O services}


\indexlibrary{\idxhdr{experimental/io_context}}%
\rSec1[io_context.synop]{Header \tcode{<experimental/io_context>} synopsis}

\begin{codeblock}
namespace std {
namespace experimental {
namespace net {
inline namespace v1 {

  class io_context;

} // inline namespace v1
} // namespace net
} // namespace experimental
} // namespace std
\end{codeblock}



\rSec1[io_context.io_context]{Class \tcode{io_context}}

\indexlibrary{\idxcode{io_context}}%
\begin{codeblock}
namespace std {
namespace experimental {
namespace net {
inline namespace v1 {

  class io_context : public execution_context
  {
  public:
    // types:

    class executor_type;
    using count_type = @\textit{implementation-defined}@;

    // construct / copy / destroy:

    io_context();
    explicit io_context(int concurrency_hint);
    io_context(const io_context&) = delete;
    io_context& operator=(const io_context&) = delete;

    // io_context operations:

    executor_type get_executor() noexcept;

    count_type run();
    template<class Rep, class Period>
      count_type run_for(const chrono::duration<Rep, Period>& rel_time);
    template<class Clock, class Duration>
      count_type run_until(const chrono::time_point<Clock, Duration>& abs_time);

    count_type run_one();
    template<class Rep, class Period>
      count_type run_one_for(const chrono::duration<Rep, Period>& rel_time);
    template<class Clock, class Duration>
      count_type run_one_until(const chrono::time_point<Clock, Duration>& abs_time);

    count_type poll();

    count_type poll_one();

    void stop();

    bool stopped() const noexcept;

    void restart();
  };

} // inline namespace v1
} // namespace net
} // namespace experimental
} // namespace std
\end{codeblock}

\pnum
The class \tcode{io_context} satisfies the ExecutionContext type requirements~(\ref{async.reqmts.executioncontext}).

\indexlibrarymember{count_type}{io_context}%
\pnum
\tcode{count_type} is an \impldef{type of \tcode{io_context::count_type}}
unsigned integral type of at least 32 bits.

\pnum
The \tcode{io_context} member functions \tcode{run}, \tcode{run_for}, \tcode{run_until}, \tcode{run_one}, \tcode{run_one_for}, \tcode{run_one_until}, \tcode{poll}, and \tcode{poll_one} are collectively referred to as the \defn{run functions}. The run functions must be called for the \tcode{io_context} to perform asynchronous operations~(\ref{defs.async.op}) on behalf of a \Cpp program. Notification that an asynchronous operation has completed is delivered by execution of the associated completion handler function object, as determined by the requirements for asynchronous operations~(\ref{async.reqmts.async}).

\pnum
For an object of type \tcode{io_context}, \defn{outstanding work} is defined as the sum of:

\begin{itemize}
\item
the total number of calls to the \tcode{on_work_started} function, less the total number of calls to the \tcode{on_work_finished} function, to any executor of the \tcode{io_context}.
\item
the number of function objects that have been added to the \tcode{io_context} via any executor of the \tcode{io_context}, but not yet executed; and
\item
the number of function objects that are currently being executed by the \tcode{io_context}.
\end{itemize}

\pnum
If at any time the outstanding work falls to \tcode{0}, the \tcode{io_context} is stopped as if by \tcode{stop()}.

\pnum
The \tcode{io_context} member functions \tcode{get_executor}, \tcode{stop}, and \tcode{stopped}, the run functions, and the \tcode{io_context::executor_type} copy constructors, member functions and comparison operators, do not introduce data races as a result of concurrent calls to those functions from different threads of execution. \begin{note} The \tcode{restart} member function is excluded from these thread safety requirements. \end{note}


\rSec2[io_context.io_context.members]{\tcode{io_context} members}

\indexlibrary{\idxcode{io_context}!constructor}%
\begin{itemdecl}
io_context();
explicit io_context(int concurrency_hint);
\end{itemdecl}

\begin{itemdescr}
\pnum
\effects Creates an object of class \tcode{io_context}.

\pnum
\remarks The \tcode{concurrency_hint} parameter is a suggestion to the implementation on the number of threads that should process asynchronous operations and execute function objects.
\end{itemdescr}

\indexlibrarymember{get_executor}{io_context}%
\begin{itemdecl}
executor_type get_executor() noexcept;
\end{itemdecl}

\begin{itemdescr}
\pnum
\returns An executor that may be used for submitting function objects to the \tcode{io_context}.
\end{itemdescr}

\indexlibrarymember{run}{io_context}%
\begin{itemdecl}
count_type run();
\end{itemdecl}

\begin{itemdescr}
\pnum
\requires Must not be called from a thread that is currently calling a run function.

\pnum
\effects Equivalent to: 
\begin{codeblock}
count_type n = 0;
while (run_one())
  if (n != numeric_limits<count_type>::max())
    ++n;
\end{codeblock}


\pnum
\returns \tcode{n}.
\end{itemdescr}

\indexlibrarymember{run_for}{io_context}%
\begin{itemdecl}
template<class Rep, class Period>
  count_type run_for(const chrono::duration<Rep, Period>& rel_time);
\end{itemdecl}

\begin{itemdescr}
\pnum
\effects Equivalent to: 
\begin{codeblock}
return run_until(chrono::steady_clock::now() + rel_time);
\end{codeblock}

\end{itemdescr}

\indexlibrarymember{run_until}{io_context}%
\begin{itemdecl}
template<class Clock, class Duration>
  count_type run_until(const chrono::time_point<Clock, Duration>& abs_time);
\end{itemdecl}

\begin{itemdescr}
\pnum
\effects Equivalent to: 
\begin{codeblock}
count_type n = 0;
while (run_one_until(abs_time))
  if (n != numeric_limits<count_type>::max())
    ++n;
\end{codeblock}


\pnum
\returns \tcode{n}.
\end{itemdescr}

\indexlibrarymember{run_one}{io_context}%
\begin{itemdecl}
count_type run_one();
\end{itemdecl}

\begin{itemdescr}
\pnum
\requires Must not be called from a thread that is currently calling a run function.

\pnum
\effects If the \tcode{io_context} object has no outstanding work, performs \tcode{stop()}. Otherwise, blocks while the io_context has outstanding work, or until the \tcode{io_context} is stopped, or until one function object has been executed.

\pnum
If an executed function object throws an exception, the exception propagates to the caller of \tcode{run_one()}. The \tcode{io_context} state is as if the function object had returned normally.

\pnum
\returns \tcode{1} if a function object was executed, otherwise \tcode{0}.

\pnum
Notes: This function may invoke additional function objects through nested calls to the \tcode{io_context} executor's \tcode{dispatch} member function. These do not count towards the return value.
\end{itemdescr}

\indexlibrarymember{run_one_for}{io_context}%
\begin{itemdecl}
template<class Rep, class Period>
  count_type run_one_for(const chrono::duration<Rep, Period>& rel_time);
\end{itemdecl}

\begin{itemdescr}
\pnum
\effects Equivalent to: 
\begin{codeblock}
return run_one_until(chrono::steady_clock::now() + rel_time);
\end{codeblock}

\end{itemdescr}

\indexlibrarymember{run_one_until}{io_context}%
\begin{itemdecl}
template<class Clock, class Duration>
  count_type run_one_until(const chrono::time_point<Clock, Duration>& abs_time);
\end{itemdecl}

\begin{itemdescr}
\pnum
\effects If the \tcode{io_context} object has no outstanding work, performs \tcode{stop()}. Otherwise, blocks while the io_context has outstanding work, or until the expiration of the absolute timeout (\CppXref{thread.req.timing}) specified by \tcode{abs_time}, or until the \tcode{io_context} is stopped, or until one function object has been executed.

\pnum
If an executed function object throws an exception, the exception propagates to the caller of \tcode{run_one()}. The \tcode{io_context} state is as if the function object had returned normally.

\pnum
\returns \tcode{1} if a function object was executed, otherwise \tcode{0}.

\pnum
Notes: This function may invoke additional function objects through nested calls to the \tcode{io_context} executor's \tcode{dispatch} member function. These do not count towards the return value.
\end{itemdescr}

\indexlibrarymember{poll}{io_context}%
\begin{itemdecl}
count_type poll();
\end{itemdecl}

\begin{itemdescr}
\pnum
\effects Equivalent to: 
\begin{codeblock}
count_type n = 0;
while (poll_one())
  if (n != numeric_limits<count_type>::max())
    ++n;
\end{codeblock}


\pnum
\returns \tcode{n}.
\end{itemdescr}

\indexlibrarymember{poll_one}{io_context}%
\begin{itemdecl}
count_type poll_one();
\end{itemdecl}

\begin{itemdescr}
\pnum
\effects If the \tcode{io_context} object has no outstanding work, performs \tcode{stop()}. Otherwise, if there is a function object ready for immediate execution, executes it.

\pnum
If an executed function object throws an exception, the exception propagates to the caller of \tcode{poll_one()}. The \tcode{io_context} state is as if the function object had returned normally.

\pnum
\returns \tcode{1} if a function object was invoked, otherwise \tcode{0}.

\pnum
Notes: This function may invoke additional function objects through nested calls to the \tcode{io_context} executor's \tcode{dispatch} member function. These do not count towards the return value.
\end{itemdescr}

\indexlibrarymember{stop}{io_context}%
\begin{itemdecl}
void stop();
\end{itemdecl}

\begin{itemdescr}
\pnum
\effects Stops the \tcode{io_context}. Concurrent calls to any run function will end as soon as possible. If a call to a run function is currently executing a function object, the call will end only after completion of that function object. The call to \tcode{stop()} returns without waiting for concurrent calls to run functions to complete.

\pnum
\postconditions \tcode{stopped() == true}.

\pnum
\begin{note} When \tcode{stopped() == true}, subsequent calls to a run function will exit immediately with a return value of \tcode{0}, without executing any function objects. An \tcode{io_context} remains in the stopped state until a call to \tcode{restart()}. \end{note}
\end{itemdescr}

\indexlibrarymember{stopped}{io_context}%
\begin{itemdecl}
bool stopped() const noexcept;
\end{itemdecl}

\begin{itemdescr}
\pnum
\returns \tcode{true} if the \tcode{io_context} is stopped.
\end{itemdescr}

\indexlibrarymember{restart}{io_context}%
\begin{itemdecl}
void restart();
\end{itemdecl}

\begin{itemdescr}
\pnum
\postconditions \tcode{stopped() == false}.
\end{itemdescr}




\rSec1[io_context.exec]{Class \tcode{io_context::executor_type}}

\indexlibrary{\idxcode{io_context::executor_type}}%
\begin{codeblock}
namespace std {
namespace experimental {
namespace net {
inline namespace v1 {

  class io_context::executor_type
  {
  public:
    // construct / copy / destroy:

    executor_type(const executor_type& other) noexcept;
    executor_type(executor_type&& other) noexcept;

    executor_type& operator=(const executor_type& other) noexcept;
    executor_type& operator=(executor_type&& other) noexcept;

    // executor operations:

    bool running_in_this_thread() const noexcept;

    io_context& context() const noexcept;

    void on_work_started() const noexcept;
    void on_work_finished() const noexcept;

    template<class Func, class ProtoAllocator>
      void dispatch(Func&& f, const ProtoAllocator& a) const;
    template<class Func, class ProtoAllocator>
      void post(Func&& f, const ProtoAllocator& a) const;
    template<class Func, class ProtoAllocator>
      void defer(Func&& f, const ProtoAllocator& a) const;
  };

  bool operator==(const io_context::executor_type& a,
                  const io_context::executor_type& b) noexcept;
  bool operator!=(const io_context::executor_type& a,
                  const io_context::executor_type& b) noexcept;

} // inline namespace v1
} // namespace net
} // namespace experimental
} // namespace std
\end{codeblock}

\pnum
\tcode{io_context::executor_type} is a type satisfying the Executor requirements~(\ref{async.reqmts.executor}). Objects of type \tcode{io_context::executor_type} are associated with an \tcode{io_context}, and function objects submitted using the \tcode{dispatch}, \tcode{post}, or \tcode{defer} member functions will be executed by the \tcode{io_context} from within a run function.]


\rSec2[io_context.exec.cons]{\tcode{io_context::executor_type} constructors}

\indexlibrary{\idxcode{io_context::executor_type}!constructor}%
\begin{itemdecl}
executor_type(const executor_type& other) noexcept;
\end{itemdecl}

\begin{itemdescr}
\pnum
\postconditions \tcode{*this == other}.
\end{itemdescr}

\begin{itemdecl}
executor_type(executor_type&& other) noexcept;
\end{itemdecl}

\begin{itemdescr}
\pnum
\postconditions \tcode{*this} is equal to the prior value of \tcode{other}.
\end{itemdescr}



\rSec2[io_context.exec.assign]{\tcode{io_context::executor_type} assignment}

\indexlibrarymember{operator=}{io_context::executor_type}%
\begin{itemdecl}
executor_type& operator=(const executor_type& other) noexcept;
\end{itemdecl}

\begin{itemdescr}
\pnum
\postconditions \tcode{*this == other}.

\pnum
\returns \tcode{*this}.
\end{itemdescr}

\begin{itemdecl}
executor_type& operator=(executor_type&& other) noexcept;
\end{itemdecl}

\begin{itemdescr}
\pnum
\postconditions \tcode{*this} is equal to the prior value of \tcode{other}.

\pnum
\returns \tcode{*this}.
\end{itemdescr}



\rSec2[io_context.exec.ops]{\tcode{io_context::executor_type} operations}

\indexlibrarymember{running_in_this_thread}{io_context::executor_type}%
\begin{itemdecl}
bool running_in_this_thread() const noexcept;
\end{itemdecl}

\begin{itemdescr}
\pnum
\returns \tcode{true} if the current thread of execution is calling a run function of the associated \tcode{io_context} object. \begin{note} That is, the current thread of execution's call chain includes a run function. \end{note}
\end{itemdescr}

\indexlibrarymember{context}{io_context::executor_type}%
\begin{itemdecl}
io_context& context() const noexcept;
\end{itemdecl}

\begin{itemdescr}
\pnum
\returns A reference to the associated \tcode{io_context} object.
\end{itemdescr}

\indexlibrarymember{on_work_started}{io_context::executor_type}%
\begin{itemdecl}
void on_work_started() const noexcept;
\end{itemdecl}

\begin{itemdescr}
\pnum
\effects Increments the count of outstanding work associated with the \tcode{io_context}.
\end{itemdescr}

\indexlibrarymember{on_work_finished}{io_context::executor_type}%
\begin{itemdecl}
void on_work_finished() const noexcept;
\end{itemdecl}

\begin{itemdescr}
\pnum
\effects Decrements the count of outstanding work associated with the \tcode{io_context}.
\end{itemdescr}

\indexlibrarymember{dispatch}{io_context::executor_type}%
\begin{itemdecl}
template<class Func, class ProtoAllocator>
  void dispatch(Func&& f, const ProtoAllocator& a) const;
\end{itemdecl}

\begin{itemdescr}
\pnum
\effects If \tcode{running_in_this_thread()} is \tcode{true}, calls \tcode{\placeholdernc{DECAY_COPY}(forward<Func>(f))()} (\CppXref{thread.decaycopy}). \begin{note} If \tcode{f} exits via an exception, the exception propagates to the caller of \tcode{dispatch()}. \end{note} Otherwise, calls \tcode{post(forward<Func>(f), a)}.
\end{itemdescr}

\indexlibrarymember{post}{io_context::executor_type}%
\begin{itemdecl}
template<class Func, class ProtoAllocator>
  void post(Func&& f, const ProtoAllocator& a) const;
\end{itemdecl}

\begin{itemdescr}
\pnum
\effects Adds \tcode{f} to the \tcode{io_context}.
\end{itemdescr}

\indexlibrarymember{defer}{io_context::executor_type}%
\begin{itemdecl}
template<class Func, class ProtoAllocator>
  void defer(Func&& f, const ProtoAllocator& a) const;
\end{itemdecl}

\begin{itemdescr}
\pnum
\effects Adds \tcode{f} to the \tcode{io_context}.
\end{itemdescr}



\rSec2[io_context.exec.comparisons]{\tcode{io_context::executor_type} comparisons}

\indexlibrarymember{operator==}{io_context::executor_type}%
\begin{itemdecl}
bool operator==(const io_context::executor_type& a,
                const io_context::executor_type& b) noexcept;
\end{itemdecl}

\begin{itemdescr}
\pnum
\returns \tcode{addressof(a.context()) == addressof(b.context())}.
\end{itemdescr}

\indexlibrarymember{operator"!=}{io_context::executor_type}%
\begin{itemdecl}
bool operator!=(const io_context::executor_type& a,
                const io_context::executor_type& b) noexcept;
\end{itemdecl}

\begin{itemdescr}
\pnum
\returns \tcode{!(a == b)}.
\end{itemdescr}




