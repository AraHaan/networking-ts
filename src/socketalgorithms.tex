%!TEX root = ts.tex

\rSec0[socket.algo]{Socket algorithms}


\rSec1[socket.algo.connect]{Synchronous connect operations}

\begin{itemdecl}
template<class Protocol, class EndpointSequence>
  typename Protocol::endpoint connect(basic_socket<Protocol>& s,
                                      const EndpointSequence& endpoints);
template<class Protocol, class InputIterator>
  typename Protocol::endpoint connect(basic_socket<Protocol>& s,
                                      const EndpointSequence& endpoints,
                                      error_code& ec);
\end{itemdecl}

\begin{itemdescr}
\pnum
\returns \tcode{connect(s, endpoints, [](auto, auto)\{ return true; \}, ec)}.
\end{itemdescr}

\begin{itemdecl}
template<class Protocol, class EndpointSequence, class ConnectCondition>
  typename Protocol::endpoint connect(basic_socket<Protocol>& s,
                                      const EndpointSequence& endpoints,
                                      ConnectCondition c);
template<class Protocol, class InputIterator, class ConnectCondition>
  typename Protocol::endpoint connect(basic_socket<Protocol>& s,
                                      const EndpointSequence& endpoints,
                                      ConnectCondition c, error_code& ec);
\end{itemdecl}

\begin{itemdescr}
\pnum
\effects Performs \tcode{ec.clear()}, then finds the first element \tcode{ep} in the sequence \tcode{endpoints} for which:
\begin{itemize}
\item
\tcode{c(ec, ep)} yields \tcode{true};
\item
\tcode{s.close(ec)} succeeds;
\item
\tcode{s.open(ep.protocol(), ec)} succeeds; and
\item
\tcode{s.connect(ep, ec)} succeeds.
\end{itemize}

\pnum
\returns \tcode{typename Protocol::endpoint()} if no such element is found, otherwise \tcode{ep}.

\pnum
\errors
\begin{itemize}
\item
\tcode{socket_errc::not_found} --- if \tcode{endpoints.empty()} or if the function object \tcode{c} returned \tcode{false} for all elements in the sequence.
\end{itemize}
\end{itemdescr}

\begin{itemdecl}
template<class Protocol, class InputIterator>
  InputIterator connect(basic_socket<Protocol>& s,
                        InputIterator first, InputIterator last);
template<class Protocol, class InputIterator>
  InputIterator connect(basic_socket<Protocol>& s,
                        InputIterator first, InputIterator last,
                        error_code& ec);
\end{itemdecl}

\begin{itemdescr}
\pnum
\returns \tcode{connect(s, first, last, [](auto, auto)\{ return true; \}, ec)}.
\end{itemdescr}

\begin{itemdecl}
template<class Protocol, class InputIterator, class ConnectCondition>
  InputIterator connect(basic_socket<Protocol>& s,
                        InputIterator first, InputIterator last,
                        ConnectCondition c);
template<class Protocol, class InputIterator, class ConnectCondition>
  InputIterator connect(basic_socket<Protocol>& s,
                        InputIterator first, InputIterator last,
                        ConnectCondition c, error_code& ec);
\end{itemdecl}

\begin{itemdescr}
\pnum
\effects Performs \tcode{ec.clear()}, then finds the first iterator \tcode{i} in the range \range{first}{last} for which:
\begin{itemize}
\item
\tcode{c(ec, *i)} yields \tcode{true};
\item
\tcode{s.close(ec)} succeeds;
\item
\tcode{s.open(typename Protocol::endpoint(*i).protocol(), ec)} succeeds; and
\item
\tcode{s.connect(*i, ec)} succeeds.
\end{itemize}

\pnum
\returns \tcode{last} if no such iterator is found, otherwise \tcode{i}.

\pnum
\errors
\begin{itemize}
\item
\tcode{socket_errc::not_found} --- if \tcode{first == last} or if the function object \tcode{c} returned \tcode{false} for all iterators in the range.
\end{itemize}
\end{itemdescr}



\rSec1[socket.algo.async.connect]{Asynchronous connect operations}

\begin{itemdecl}
template<class Protocol, class EndpointSequence, class CompletionToken>
  @\DEDUCED@ async_connect(basic_socket<Protocol>& s,
                        const EndpointSequence& endpoints,
                        CompletionToken&& token);
\end{itemdecl}

\begin{itemdescr}
\pnum
\returns \tcode{async_connect(s, endpoints, [](auto, auto)\{ return true; \}, forward<Completion\-Token>(token))}.
\end{itemdescr}

\begin{itemdecl}
template<class Protocol, class InputIterator,
  class ConnectCondition, class CompletionToken>
    @\DEDUCED@ async_connect(basic_socket<Protocol>& s,
                          const EndpointSequence& endpoints,
                          ConnectCondition c,
                          CompletionToken&& token);
\end{itemdecl}

\begin{itemdescr}
\pnum
\completionsig \tcode{void(error_code ec, typename Protocol::endpoint ep)}.

\pnum
\effects Performs \tcode{ec.clear()}, then finds the first element \tcode{ep} in the sequence \tcode{endpoints} for which:
\begin{itemize}
\item
\tcode{c(ec, ep)} yields \tcode{true};
\item
\tcode{s.close(ec)} succeeds;
\item
\tcode{s.open(ep.protocol(), ec)} succeeds; and
\item
 the asynchronous operation \tcode{s.async_connect(ep, unspecified)} succeeds.\tcode{ec} is updated with the result of the \tcode{s.async_connect(ep, unspecified)} operation, if any. If no such element is found, or if the operation fails with one of the error conditions listed below, \tcode{ep} is set to \tcode{typename Protocol::endpoint()}. \enternote The underlying \tcode{close}, \tcode{open}, and \tcode{async_connect} operations are performed sequentially. \exitnote
\end{itemize}

\pnum
\errors
\begin{itemize}
\item
\tcode{socket_errc::not_found} --- if \tcode{endpoints.empty()} or if the function object \tcode{c} returned \tcode{false} for all elements in the sequence.
\item
\tcode{errc::operation_canceled} --- if \tcode{s.is_open() == false} immediately following an \tcode{async_connect} operation on the underlying socket.
\end{itemize}
\end{itemdescr}

\begin{itemdecl}
template<class Protocol, class InputIterator, class CompletionToken>
  @\DEDUCED@ async_connect(basic_socket<Protocol>& s,
                        InputIterator first, InputIterator last,
                        CompletionToken&& token);
\end{itemdecl}

\begin{itemdescr}
\pnum
\returns \tcode{async_connect(s, first, last, [](auto, auto)\{ return true; \}, forward<Completion\-Token>(token))}.
\end{itemdescr}

\begin{itemdecl}
template<class Protocol, class InputIterator,
  class ConnectCondition, class CompletionToken>
    @\DEDUCED@ async_connect(basic_socket<Protocol>& s,
                          InputIterator first, InputIterator last,
                          ConnectCondition c,
                          CompletionToken&& token);
\end{itemdecl}

\begin{itemdescr}
\pnum
\completionsig \tcode{void(error_code ec, InputIterator i)}.

\pnum
\effects Performs \tcode{ec.clear()}, then finds the first iterator \tcode{i} in the range [\tcode{first},\tcode{last}) for which:
\begin{itemize}
\item
\tcode{c(ec, *i)} yields \tcode{true};
\item
\tcode{s.close(ec)} succeeds;
\item
\tcode{s.open(typename Protocol::endpoint(*i).protocol(), ec)} succeeds; and
\item
 the asynchronous operation \tcode{s.async_connect(*i, unspecified)} succeeds.
\end{itemize}
\pnum
\tcode{ec} is updated with the result of the \tcode{s.async_connect(*i, unspecified)} operation, if any. If no such iterator is found, or if the operation fails with one of the error conditions listed below, \tcode{i} is set to \tcode{last}. \enternote The underlying \tcode{close}, \tcode{open}, and \tcode{async_connect} operations are performed sequentially. \exitnote

\pnum
\errors
\begin{itemize}
\item
\tcode{socket_errc::not_found} --- if \tcode{first == last} or if the function object \tcode{c} returned \tcode{false} for all iterators in the range.
\item
\tcode{errc::operation_canceled} --- if \tcode{s.is_open() == false} immediately following an \tcode{async_connect} operation on the underlying socket.
\end{itemize}
\end{itemdescr}



